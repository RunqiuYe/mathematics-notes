\documentclass[a4paper]{article}
\usepackage{parskip}

\def\nterm {Spring}
\def\nyear {2025}
\def\ncourse {Mathematical Studies Analysis}

\input{../header}

\renewcommand{\seqinfl}[1]{\{ #1 \}_{l=0}^\infty}
\renewcommand{\cupinfj}{\bigcup_{j=0}^\infty}
\renewcommand{\capinfj}{\bigcap_{j=0}^\infty}
\renewcommand{\suminfj}{\sum_{j=0}^\infty}
\renewcommand{\seqinfj}[1]{\{ #1 \}_{j=0}^\infty}
\renewcommand{\cupinfi}{\bigcup_{i=0}^\infty}
\renewcommand{\capinfi}{\bigcap_{i=0}^\infty}
\renewcommand{\suminfi}{\sum_{i=0}^\infty}
\renewcommand{\seqinfi}[1]{\{ #1 \}_{i=0}^\infty}
\renewcommand{\cupinfn}{\bigcup_{n=0}^\infty}
\renewcommand{\capinfn}{\bigcap_{n=0}^\infty}
\renewcommand{\suminfn}{\sum_{n=0}^\infty}
\renewcommand{\seqinfn}[1]{\{ #1 \}_{n=0}^\infty}
\renewcommand{\cupinfk}{\bigcup_{k=0}^\infty}
\renewcommand{\sqcupinfk}{\bigsqcup_{k=0}^\infty}
\renewcommand{\capinfk}{\bigcap_{k=0}^\infty}
\renewcommand{\suminfk}{\sum_{k=0}^\infty}
\renewcommand{\seqinfk}[1]{\{ #1 \}_{k=0}^\infty}
\renewcommand{\cupinfm}{\bigcup_{m=0}^\infty}
\renewcommand{\capinfm}{\bigcap_{m=0}^\infty}
\renewcommand{\suminfm}{\sum_{m=0}^\infty}
\renewcommand{\seqinfm}[1]{\{ #1 \}_{m=0}^\infty}

\renewcommand{\cupj}{\bigcup_{j=0}}
\renewcommand{\capj}{\bigcap_{j=0}}
\renewcommand{\sumj}{\sum_{j=0}}
\renewcommand{\seqj}[1]{\{ #1 \}_{j=0}}
\renewcommand{\cupi}{\bigcup_{i=0}}
\renewcommand{\capi}{\bigcap_{i=0}}
\renewcommand{\sumi}{\sum_{i=0}}
\renewcommand{\seqi}[1]{\{ #1 \}_{i=0}}
\renewcommand{\cupn}{\bigcup_{n=0}}
\renewcommand{\capn}{\bigcap_{n=0}}
\renewcommand{\sumn}{\sum_{n=0}}
\renewcommand{\seqn}[1]{\{ #1 \}_{n=0}}
\renewcommand{\cupk}{\bigcup_{k=0}}
\renewcommand{\capk}{\bigcap_{k=0}}
\renewcommand{\sumk}{\sum_{k=0}}
\renewcommand{\seqk}[1]{\{ #1 \}_{k=0}}
\renewcommand{\cupm}{\bigcup_{m=0}}
\renewcommand{\capm}{\bigcap_{m=0}}
\renewcommand{\summ}{\sum_{m=0}}
\renewcommand{\seqm}[1]{\{ #1 \}_{m=0}}

\begin{document}

\section*{Practice problems A}
\subsubsection*{Problem 1}
Suppose $\omega: [0, \infty) \to [0, \infty]$ any function 
such that $\omega(x) = 0$ if and only if $x = 0$, 
$\omega$ continuous at 
$0$, and $\omega$ is nondecreasing. For $f: X \to Z$ define
\[
[f]_\omega = \sup\left\{ \frac{d(f(x),f(y))}{\omega(d(x,y))} :
x, y \in X, x \neq y \right\}
\]
and the space 
\[
C^{0, \omega} (X; Z) = \left\{ f: X \to Z \mid [f]_\omega < \infty \right\}.
\]
{
\newcommand{\cw}{{C^{0, \omega}}}
\newcommand{\cwb}{{C^{0, \omega}_b}}
\begin{enumerate}
  \item Prove that $\cw(X; Z) \subset C^0(X; Z)$.

  \item Suppose $Z$ Banach. Show that $\norm{f}_{\cw} = \norm{f}_{C^0} + [f]_\omega$
  is a norm on $\cwb(X; Z) = C^0_b(X; Z) \cap \cw(X; Z)$, and 
  that $\cwb(X;Z)$ is complete with respect to this norm.

  \item Suppose that $X$ is compact and $d \in \N$, show that 
  $B_{C^{0, \omega}(X; \R^d)} [0, 1] \subset C^0(X; \R^d)$
  is compact.

  \item Suppose $X$ compact and infinte, and $d \in \N$.
  Show that $B_{\cw}[0,1] \subset \cw$ is not compact.
  Conclude that $\id: (\cw, \norm{\cdot}_{C^0}) \to (\cw, \norm{\cdot}_{\cw})$
  is not continuous. Also conclude that $(\cw, \norm{\cdot}_{C^0})$ 
  is not complete.

  \item Another way to see this last fact is to first prove 
  $\cw(X; \R^d)$ is a strict subset of $C^0(X; \R^d)$. It is helpful 
  to study the sets $E_n = \left\{ f \in C^0(X; \R^d) : [f]_\omega \leq n \right\}$.
  Show that $\cw(X; \R)$ is dense in $C^0(X; \R)$. 
  Use this to show that $\cw(X; \R^d)$ is dense in $C^0(X; \R)$,
  and conclude $(\cw(X; \R^d), \norm{\cdot}_{C^0})$ 
  is not complete.
\end{enumerate}


\begin{proof}
\begin{enumerate}
\item Let $x \in X$ and $\epsilon > 0$. It follws that for 
any $x \neq y$ we have 
\[
d(f(x), f(y)) \leq [f]_\omega \omega(d(x, y)).
\]
Since $\omega(0) = 0$, $\omega$ continuous
at $0$, and $\omega$ is nondecreasing, we can find 
$\delta > 0$ such that $0 \leq t < \delta$ implies 
$0 \leq \omega(t) < \epsilon$. 
Therefore,
$d(x, y) < \delta$ implies $d(f(x), f(y)) < \epsilon [f]_{\omega}$.
Since $[f]_{\omega} < \infty$, $f$ is continuous and 
$\cw (X; Z) \subset C^0(X; Z)$.

\item It is easy to show that $\norm{\cdot}_{\cw}$ is indeed a
norm on $\cwb(X; Z)$. Now we show that $\cwb(X; Z)$
is complete with respect to this norm.
Suppose $\left\{ f_n \right\} \subset \cwb$ Cauchy. Then 
it is also Cauchy in $C^0_b$. Therefore there is $f \in C^0_b$
such that $f_n \to f$ under $C^0$ norm. Remain to show 
$[f - f_n]_\omega \to 0$. Let $x, y \in X$ and $x \neq y$
and $m, n \geq N$ implies $[f_m - f_n]_\omega < \epsilon$.
Then, 
\[
\begin{aligned}
  \frac{\norm{f_m(x) - f_m(y) - f_n(x) + f_n(y)}_Z}{\omega(d(x, y))}
  < \epsilon.
\end{aligned}
\]
Take $m \to \infty$ and take supremum of all 
$x, y \in X$ with $x \neq y$
completes the proof.


\end{enumerate}
\end{proof}
}
\end{document}