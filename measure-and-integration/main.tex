\documentclass[a4paper]{article}
\usepackage{parskip}
\usepackage{lipsum}

\def\nterm {Summer}
\def\nyear {2025}
\def\ncourse {Measure and Integration}

\input{../header}

\newcommand{\TODO}{\textcolor{red}{\textbf{*** TO-DO ***}}}

\begin{document}
\maketitle

\tableofcontents

\setcounter{section}{2}
\section{Signed measure and differentiation}

\begin{ex}[Folland 3.18]
Let $\nu$ be a complex measure on $(X, \mf)$. Prove that 
$L^1 (\nu) = L^1 (\abs{\nu})$ and if $f \in L^1 (\nu)$, then 
$\abs{\int f \d \nu} \leq \int \abs{f} \d \abs{\nu}$.
\end{ex}

\begin{proof}
For $L^1 (\nu) \subset L^1 (\abs{\nu})$, consider $f \in L^1 (\nu)$.
Note that $\nu = \nu_r + i \nu_i$ and it is easy to 
verify that $\abs{i \nu_i} = \abs{\nu_i}$.
Therefore by Proposition 3.14, we have $\abs{\nu} \leq 
\abs{\nu_r} + \abs{\nu_i}$. It follows that 
\[
\begin{aligned}
\int \abs{f} \d \abs{\nu} 
&\leq \int \abs{f} \d \abs{\nu_r} + \int \abs{f} \d
\abs{\nu_i} \\
&= \int \abs{f} \d \nu_r^+ + \int \abs{f} \d \nu_r^- 
+ \int \abs{f} \d \nu_i^+ + \int \abs{f} \d \nu_i^-.
\end{aligned}
\]
Since $f \in L^1 (\nu)$, all four terms are finite and thus 
$f \in L^1 (\abs{\nu})$. 

For $L^1 (\abs{\nu}) \subset L^1 (\nu)$, consider $f \in L^1 
(\abs{\nu})$. Then we have 
\[
\int \abs{f} \d \nu 
= \int \abs{f} \frac{d \nu}{d \abs{\nu}} \d \abs{\nu} 
\leq \int \abs{f} \abs{\frac{d \nu}{d \abs{\nu}}} \d \abs{\nu} 
= \int \abs{f} \d \abs{\nu},
\]
where we have used the fact that $d \nu / d \abs{\nu}$ has 
absolute value $1$ $\abs{\nu}$-a.e. This shows that 
$f \in L^1 (\nu)$. 

Moreover, we have that $\abs{\int f \d \nu} 
\leq \int \abs{f} \d \nu$. Therefore,
\[
\abs{\int f \d \nu} \leq \int \abs{f} \d \abs{\nu},
\]
as desired.
\end{proof}

\begin{ex}[Folland 3.19]
If $\nu, \mu$ are complex measures and $\lambda$ 
is a positive measure, then $\nu \perp \mu$ if and only 
if $\abs{\nu} \perp \abs{\mu}$, and $\nu \ll \lambda$ 
if and only if $\abs{\nu} \ll \lambda$.
\end{ex}

\begin{proof}
\TODO
\end{proof}

\begin{ex}[Folland 3.20]
If $\nu$ is a complex measure on $(X, \mf)$ and 
$\nu(X) = \abs{\nu}(X)$, then $\nu = \abs{\nu}$.
\end{ex}

\begin{proof}
By Lebesgue-Randon-Nikodym theorem, we have 
$d \nu = f \d \mu$ for some function $f$ and positive 
measure $\mu$. It follows that $d \abs{\nu} 
= \abs{f} \d \mu$ and 
\[
\int f \d \mu = \int \abs{f} \d \mu.
\]
This shows that $f = \abs{f}$ $\mu$-a.e., and thus 
$\nu = \abs{\nu}$.
\end{proof}

\end{document}