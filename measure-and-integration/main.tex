\documentclass[a4paper]{article}
\usepackage{parskip}
\usepackage{lipsum}

\def\nterm {Summer}
\def\nyear {2025}
\def\ncourse {Measure and Integration}

\input{../header}

\newcommand{\TODO}{\textcolor{red}{\textbf{*** TO-DO ***}}}

\begin{document}
\maketitle

\tableofcontents

\setcounter{section}{2}
\section{Signed measure and differentiation}

\begin{ex}[Folland 3.2]
If $\nu$ a signed measure, $E$ is $\nu$-null if and only if 
$\abs{\nu}(E) = 0$. Also if $\nu$ and $\mu$ are signed measures,
then $\nu \perp \mu$ if and only if $\abs{\nu} \perp \mu$ 
if and only if $\nu^+ \perp \mu$ and $\nu^- \perp \mu$.
\end{ex}

\begin{proof}
It is clear that $\abs{\nu}(E) = 0$ implies that $E$ is $\nu$-null.
Now suppose $E$ is $\nu$-null and let $X = P \cup N$ be the 
Hahn decomposition of $\nu$. Suppose for contradiction 
that $\abs{\nu}(E) > 0$, then it follows that 
$\nu^+(E) > 0$ and $\nu^- (E) > 0$. We then have 
$\nu^+(E \cap P) = \nu^+ (E \cap P) + \nu^+ (E \cap N) = \nu^+(E)
> 0$, but $\nu^- (E \cap P) \leq \nu^- (P) = 0$. Therefore, 
$\nu(E \cap P) > 0$, a contradiction with $E$ being $\nu$-null.
Therefore, $\abs{\nu}(E) = 0$.

Suppose $\nu \perp \mu$, then there is $X = E \cup F$ such that 
$E$ is $\nu$-null and $F$ is $\mu$-null. It follows that 
$\abs{\nu}(E) = 0$, so $\abs{\nu} \perp \mu$. Therefore, 
$\nu \perp \mu$ implies $\abs{\nu} \perp \mu$. It is clear 
that $\abs{\nu} \perp \mu$ implies $\nu^+ \perp \mu$
and $\nu^- \perp \mu$. Now suppose 
$\nu^+ \perp \mu$ and $\nu^- \perp \mu$. Then we have 
$X = E^+ \cup F^+ = E^- \cup F^-$ where $F^+$ is $\nu^+$ null,
$F^-$ is $\nu^-$ null, and $E^\pm$ is $\mu$-null. Let 
$E = E^+ \cup E^-$ and $F = F^+ \cap F^- = E^c$. Then we can 
verify that $E$ is $\mu$-null and $F$ is $\nu$-null. Therefore, 
$\nu \perp \mu$ and the proof is complete.
\end{proof}

\begin{ex}[Folland 3.3]
Let $\nu$ be a signed measure on $(X, \mf)$. 
\begin{enumerate}
  \item $L^1 (\nu) = L^1(\abs{\nu})$.
  \item If $f \in L^1 (\nu)$, then $\abs{\int f \d \nu}
  \leq \int \abs{f} \d \abs{\nu}$.
  \item If $E \in \mf$, $\abs{\nu}(E) = \sup 
  \left\{ \abs{\int_E f \d \nu} : \abs{f} \leq 1 \right\}$
\end{enumerate}  
\end{ex}

\begin{proof}
\begin{enumerate}
\item Since we have 
\[
\int \abs{f} \d \nu = \int \abs{f} \d \nu^+ - \int \abs{f} \d \nu^-, 
\quad 
\int \abs{f} \d \abs{\nu} = \int \abs{f} \d \nu^+ + \int \abs{f} 
\d \nu^-,
\]
it follows immediately that $L^1 (\nu) = L^1 (\abs{\nu})$.

\item We have 
\[
\begin{aligned}
\abs{\int f \d \nu} = \abs{\int f d \nu^+ - \int f \d \nu^-} 
\leq \int \abs{f} \d \nu^+ + \int f \d \nu^- 
= \int \abs{f} \d \abs{\nu}.
\end{aligned}
\]

\item By the previous item, we know
\[
\abs{\nu} (E) = \int_E 1 \d \abs{\nu} \geq \int_E \abs{f} 
\d \abs{\nu} \geq \abs{\int f \d \nu}.
\]
for any $\abs{f} \leq 1$. Also, $\abs{\nu}(E) = 
\int \abs{\chi_P + \chi_N} \d \nu$, where $X = P \cup N$ is the 
Hahn decomposition so $\abs{\chi_P + \chi_N} = 1$.
\end{enumerate}
\end{proof}

\begin{ex}[Folland 3.4]
If $\nu$ is a signed measure and $\lambda, \mu$ are positive
measures such that $\nu = \lambda - \mu$, then $\lambda \geq \nu^+$
and $\mu \geq \nu^-$.
\end{ex}

\begin{proof}
Let $\nu = \nu^+ - \nu^-$ be the unique decomposition. 
We then have $\lambda - \nu^+ = \mu - \nu^-$. Pick a set $E \in \mf$
such that $\nu^- (E) = 0$. We then have 
\[
\lambda (E) - \nu^+ (E) = \mu (E) \geq 0.
\]
Therefore $\lambda (E) \geq \nu^+ (E)$. On the other hand,
for $E \in \mf$ such that $\nu^+ (E) = 0$, we have 
$\lambda (E) \geq 0 = \nu^+ (E)$. In light of the Hahn 
decomposition for $\nu$ and the additivity of measure, 
we can conclude that $\lambda \geq \nu^+$, and thus
$\mu \geq \nu^-$.
\end{proof}


\begin{ex}[Folland 3.7]
Suppose $\nu$ is a signed measure on $(X, \mf)$ and $E \in \mf$.
\begin{enumerate}
\item $\nu^+ (E) = \sup \left\{ \nu(F) : F \subset E, F \in \mf 
\right\}$ and $\nu^-(E) = - \inf \left\{ \nu(F) : F \subset E, 
F \in \mf \right\}$. 

\item $\abs{\nu}(E) = \sup \left\{ \sumj^n \abs{\nu(E_j)} 
: \text{$n \in \N$, $E_1, \dots, E_n$ disjoint, and 
$\cupj^n E_j = E$} \right\}$.
\end{enumerate}
\end{ex}

\begin{proof}
\begin{enumerate}
\item First let $F \subset E$, we have $\nu(F) = \nu^+(F) - \nu^-(F)$.
It follows that $\nu(F) \leq \nu^+ (F) \leq \nu^+ (E)$. Let 
$X = P \cup N$ be the Hahn decomposition for $\nu$, then we have 
$\nu^+(E) = \nu(E \cap P)$. Similarly for $\nu^- (E)$. 

\item Let $E_1, \dots, E_n$ be disjoint and $\cupj^n E_j = E$. 
It follows that 
\[
\abs{\nu}(E) = \sumj^n \abs{\nu}(E_j) \geq \sumj^n \abs{\nu(E_j)}.
\]
Let $X = P \cup N$ be the Hahn decomposition for $\nu$, we 
then have 
\[
\abs{\nu}(E) = \nu^+ (E) + \nu^-(E) = \nu^+ (E \cap P) 
+ \nu^- (E \cap N) = \abs{\nu (E \cap P)} + \abs{\nu (E \cap N)}.
\]
This completes the proof.
\end{enumerate}
\end{proof}

\begin{ex}[Folland 3.8]
$\nu \ll \mu$ if and only if $\abs{\nu} \ll \mu$ 
if and only if $\nu^+ \ll \mu$ and $\nu^- \ll \mu$.
\end{ex}

\begin{proof}
It is clear that $\abs{\nu} \ll \mu$ implies 
$\nu^+ \ll \mu$ and $\nu^- \ll \mu$ implies 
$\nu \ll \mu$. It then remains to show that $\nu \ll \mu$ 
implies $\abs{\nu} \ll \mu$. Let $X = P \cup N$ be the Hahn 
decomposition for $\nu$ and $E \in \mf$ be such that 
$\mu(E) = 0$. It follows that 
$\mu(E \cap P) = 0$ and $\mu (E \cap N) = 0$. Then 
$\nu(E \cap P) = \nu^+ (E \cap P) = 0$ and 
$\nu(E \cap N) = \nu^- (E \cap N) = 0$. Therefore, 
\[
\nu (E) = \nu^+ (E \cap P) + \nu^- (E \cap N) = 0,
\]
as desired.
\end{proof}


\begin{ex}[Folland 3.18]
Let $\nu$ be a complex measure on $(X, \mf)$. Prove that 
$L^1 (\nu) = L^1 (\abs{\nu})$ and if $f \in L^1 (\nu)$, then 
$\abs{\int f \d \nu} \leq \int \abs{f} \d \abs{\nu}$.
\end{ex}

\begin{proof}
For $L^1 (\nu) \subset L^1 (\abs{\nu})$, consider $f \in L^1 (\nu)$.
Note that $\nu = \nu_r + i \nu_i$ and it is easy to 
verify that $\abs{i \nu_i} = \abs{\nu_i}$.
Therefore by Proposition 3.14, we have $\abs{\nu} \leq 
\abs{\nu_r} + \abs{\nu_i}$. It follows that 
\[
\begin{aligned}
\int \abs{f} \d \abs{\nu} 
&\leq \int \abs{f} \d \abs{\nu_r} + \int \abs{f} \d
\abs{\nu_i} \\
&= \int \abs{f} \d \nu_r^+ + \int \abs{f} \d \nu_r^- 
+ \int \abs{f} \d \nu_i^+ + \int \abs{f} \d \nu_i^-.
\end{aligned}
\]
Since $f \in L^1 (\nu)$, all four terms are finite and thus 
$f \in L^1 (\abs{\nu})$. 

For $L^1 (\abs{\nu}) \subset L^1 (\nu)$, consider $f \in L^1 
(\abs{\nu})$. Then we have 
\[
\int \abs{f} \d \nu 
= \int \abs{f} \frac{d \nu}{d \abs{\nu}} \d \abs{\nu} 
\leq \int \abs{f} \abs{\frac{d \nu}{d \abs{\nu}}} \d \abs{\nu} 
= \int \abs{f} \d \abs{\nu},
\]
where we have used the fact that $d \nu / d \abs{\nu}$ has 
absolute value $1$ $\abs{\nu}$-a.e. This shows that 
$f \in L^1 (\nu)$. 

Moreover, we have that $\abs{\int f \d \nu} 
\leq \int \abs{f} \d \nu$. Therefore,
\[
\abs{\int f \d \nu} \leq \int \abs{f} \d \abs{\nu},
\]
as desired.
\end{proof}

\begin{ex}[Folland 3.19]
If $\nu, \mu$ are complex measures and $\lambda$ 
is a positive measure, then $\nu \perp \mu$ if and only 
if $\abs{\nu} \perp \abs{\mu}$, and $\nu \ll \lambda$ 
if and only if $\abs{\nu} \ll \lambda$.
\end{ex}

\begin{proof}
The ``if'' direction is clear for both propositions.

Suppose $\nu \perp \mu$. Then $\nu_r \perp \mu_r$,
$\nu_i \perp \mu_r$, $\nu_r \perp \mu_i$, and $\nu_i \perp \mu_i$.
It follows that $\abs{\nu_r} \perp \mu_r$ and $\abs{\nu_i}
\perp \mu_r$. Since $\abs{\nu} \leq \abs{\nu_r} + \abs{\nu_i}$, 
we have $\abs{\nu} \perp \mu_r$. Similarly $\abs{\nu} \perp \mu_i$.
Following the same reasoning, we obtain 
$\abs{\nu} \perp \abs{\mu}$, as desired.

Suppose $\nu \ll \lambda$. 
Since $\nu = \nu_r + i \nu_i$, we have 
$\nu_r \ll \lambda$ and $\nu_i \ll \lambda$. 
Recall from Exercise 3.8 that this implies $\abs{\nu_r} \ll 
\lambda$ and $\abs{\nu_i} \ll \lambda$. Moreover, 
$\abs{\nu} \leq \abs{\nu_r} + \abs{\nu_i}$, so 
$\abs{\nu} \ll \lambda$.
\end{proof}

\begin{ex}[Folland 3.20]
If $\nu$ is a complex measure on $(X, \mf)$ and 
$\nu(X) = \abs{\nu}(X)$, then $\nu = \abs{\nu}$.
\end{ex}

\begin{proof}
By Lebesgue-Randon-Nikodym theorem, we have 
$d \nu = f \d \mu$ for some function $f$ and positive 
measure $\mu$. It follows that $d \abs{\nu} 
= \abs{f} \d \mu$ and 
\[
\int f \d \mu = \int \abs{f} \d \mu.
\]
Now let $E \in \mf$. We then have 
\[
\int_E f \d \mu + \int_{E^c} f \d \mu = 
\int_E \abs{f} \d \mu + \int_{E^c} \abs{f} \d \mu.
\]
It follows that 
\[
0 \leq \int_{E} \abs{f} - \Re (f) \d \mu
= \int_{E^c} \Re (f) - \abs{f} \d \mu \leq 0.
\]
Since $E \in \mf$ is arbitrary, 
$\Re (f) = \abs{f}$ and $\Im (f) = 0$ a.e. It follows that 
$d \nu = f \d \mu = \abs{f} \d \mu = d \abs{\nu}$.
\end{proof}

\end{document}