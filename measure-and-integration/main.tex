\documentclass[a4paper]{article}
\usepackage{parskip}
\usepackage{lipsum}

\def\nterm {Summer}
\def\nyear {2025}
\def\ncourse {Measure and Integration}

\makeatletter
\ifx \nauthor\undefined
  \def\nauthor{Runqiu Ye}
\else
\fi

\author{Notes taken by \nauthor\vspace{5pt}\\ 
Carnegie Mellon University}
\date{\nterm\ \nyear}

\usepackage{alltt}
\usepackage{amsfonts}
\usepackage{amsmath}
\usepackage{amssymb}
\usepackage{amsthm}
\usepackage{booktabs}
\usepackage{caption}
\usepackage{enumitem}
\usepackage{fancyhdr}
\usepackage{graphicx}
\usepackage{mathdots}
\usepackage{mathtools}
\usepackage{microtype}
\usepackage{multirow}
\usepackage{pdflscape}
\usepackage{pgfplots}
\usepackage{siunitx}
\usepackage{slashed}
\usepackage{tabularx}
\usepackage{tikz}
\usepackage{tkz-euclide}
\usepackage[normalem]{ulem}
\usepackage[all]{xy}
\usepackage{imakeidx}
\usepackage[includehead,includefoot,heightrounded,left=1.25in,right=1.25in]{geometry}
\usepackage{mathrsfs}
\usepackage{stmaryrd}

\makeindex[intoc, title=Index]
\indexsetup{othercode={\lhead{\emph{Index}}}}

\ifx \nextra \undefined
  \usepackage[pdftex,
    hidelinks,
    pdfauthor={Dexter Chua},
    pdfsubject={\ncourse},
    pdftitle={\ncourse},
  pdfkeywords={Cambridge Mathematics Maths Math \nterm\ \nyear\ \ncourse}]{hyperref}
  \title{\ncourse}
\else
  \usepackage[pdftex,
    hidelinks,
    pdfauthor={Dexter Chua},
    pdfsubject={Cambridge Maths Notes: \ncourse\ (\nextra)},
    pdftitle={\ncourse\ (\nextra)},
  pdfkeywords={Cambridge Mathematics Maths Math \nterm\ \nyear\ \ncourse\ \nextra}]{hyperref}

  \title{\ncourse \\ {\Large \nextra}}
  \renewcommand\printindex{}
\fi

\pgfplotsset{compat=1.12}

\pagestyle{fancyplain}
\ifx \ncoursehead \undefined
\def\ncoursehead{\ncourse}
\fi

\lhead{\emph{\nouppercase{\leftmark}}}
\ifx \nextra \undefined
  \rhead{
    \ifnum\thepage=1
    \else
      \ncoursehead
    \fi}
\else
  \rhead{
    \ifnum\thepage=1
    \else
      \ncoursehead \ (\nextra)
    \fi}
\fi
\usetikzlibrary{arrows.meta}
\usetikzlibrary{decorations.markings}
\usetikzlibrary{decorations.pathmorphing}
\usetikzlibrary{positioning}
\usetikzlibrary{fadings}
\usetikzlibrary{intersections}
\usetikzlibrary{cd}

\newcommand*{\Cdot}{{\raisebox{-0.25ex}{\scalebox{1.5}{$\cdot$}}}}
% \newcommand {\pd}[2][ ]{
%   \ifx #1 { }
%     \frac{\partial}{\partial #2}
%   \else
%     \frac{\partial^{#1}}{\partial #2^{#1}}
%   \fi
% }
\newcommand{\pd}[2]{\frac{\partial #1}{\partial #2}}
\ifx \nhtml \undefined
\else
  \renewcommand\printindex{}
  \DisableLigatures[f]{family = *}
  \let\Contentsline\contentsline
  \renewcommand\contentsline[3]{\Contentsline{#1}{#2}{}}
  \renewcommand{\@dotsep}{10000}
  \newlength\currentparindent
  \setlength\currentparindent\parindent

  \newcommand\@minipagerestore{\setlength{\parindent}{\currentparindent}}
  \usepackage[active,tightpage,pdftex]{preview}
  \renewcommand{\PreviewBorder}{0.1cm}

  \newenvironment{stretchpage}%
  {\begin{preview}\begin{minipage}{\hsize}}%
    {\end{minipage}\end{preview}}
  \AtBeginDocument{\begin{stretchpage}}
  \AtEndDocument{\end{stretchpage}}

  \newcommand{\@@newpage}{\end{stretchpage}\begin{stretchpage}}

  \let\@real@section\section
  \renewcommand{\section}{\@@newpage\@real@section}
  \let\@real@subsection\subsection
  \renewcommand{\subsection}{\@ifstar{\@real@subsection*}{\@@newpage\@real@subsection}}
\fi
\ifx \ntrim \undefined
\else
  \usepackage{geometry}
  \geometry{
    papersize={379pt, 699pt},
    textwidth=345pt,
    textheight=596pt,
    left=17pt,
    top=54pt,
    right=17pt
  }
\fi

\ifx \nisofficial \undefined
\let\@real@maketitle\maketitle
\renewcommand{\maketitle}{\@real@maketitle}
\else
\fi

% Theorems
\theoremstyle{definition}
\newtheorem*{aim}{Aim}
\newtheorem*{axiom}{Axiom}
\newtheorem*{claim}{Claim}
\newtheorem*{cor}{Corollary}
\newtheorem*{conjecture}{Conjecture}
\newtheorem*{defi}{Definition}
\newtheorem*{eg}{Example}
\newtheorem*{ex}{Exercise}
\newtheorem*{fact}{Fact}
\newtheorem*{law}{Law}
\newtheorem*{lemma}{Lemma}
\newtheorem*{notation}{Notation}
\newtheorem*{prop}{Proposition}
\newtheorem*{question}{Question}
\newtheorem*{rrule}{Rule}
\newtheorem*{thm}{Theorem}
\newtheorem*{assumption}{Assumption}

\newtheorem*{remark}{Remark}
\newtheorem*{warning}{Warning}
\newtheorem*{exercise}{Exercise}

\newtheorem{nthm}{Theorem}[section]
\newtheorem{nlemma}[nthm]{Lemma}
\newtheorem{nprop}[nthm]{Proposition}
\newtheorem{ncor}[nthm]{Corollary}


\renewcommand{\labelitemi}{--}
\renewcommand{\labelitemii}{$\circ$}
% \renewcommand{\labelenumi}{(\roman{*})}

\let\stdsection\section
\renewcommand\section{\newpage\stdsection}

\newcommand\qedsym{\hfill\ensuremath{\square}}
% Strike through
\def\st{\bgroup \ULdepth=-.55ex \ULset}



%%%%%%%%%%%%%%%%%%%%%%%%%
%%%%% Maths Symbols %%%%%
%%%%%%%%%%%%%%%%%%%%%%%%%

\newcommand{\seqinfl}[1]{\{ #1 \}_{l=1}^\infty}
\newcommand{\cupinfj}{\bigcup_{j=1}^\infty}
\newcommand{\capinfj}{\bigcap_{j=1}^\infty}
\newcommand{\suminfj}{\sum_{j=1}^\infty}
\newcommand{\seqinfj}[1]{\{ #1 \}_{j=1}^\infty}
\newcommand{\cupinfi}{\bigcup_{i=1}^\infty}
\newcommand{\capinfi}{\bigcap_{i=1}^\infty}
\newcommand{\suminfi}{\sum_{i=1}^\infty}
\newcommand{\seqinfi}[1]{\{ #1 \}_{i=1}^\infty}
\newcommand{\cupinfn}{\bigcup_{n=1}^\infty}
\newcommand{\capinfn}{\bigcap_{n=1}^\infty}
\newcommand{\suminfn}{\sum_{n=1}^\infty}
\newcommand{\seqinfn}[1]{\{ #1 \}_{n=1}^\infty}
\newcommand{\cupinfk}{\bigcup_{k=1}^\infty}
\newcommand{\sqcupinfk}{\bigsqcup_{k=1}^\infty}
\newcommand{\capinfk}{\bigcap_{k=1}^\infty}
\newcommand{\suminfk}{\sum_{k=1}^\infty}
\newcommand{\seqinfk}[1]{\{ #1 \}_{k=1}^\infty}
\newcommand{\cupinfm}{\bigcup_{m=1}^\infty}
\newcommand{\capinfm}{\bigcap_{m=1}^\infty}
\newcommand{\suminfm}{\sum_{m=1}^\infty}
\newcommand{\seqinfm}[1]{\{ #1 \}_{m=1}^\infty}

\newcommand{\cupj}{\bigcup_{j=1}}
\newcommand{\capj}{\bigcap_{j=1}}
\newcommand{\sumj}{\sum_{j=1}}
\newcommand{\seqj}[1]{\{ #1 \}_{j=1}}
\newcommand{\cupi}{\bigcup_{i=1}}
\newcommand{\capi}{\bigcap_{i=1}}
\newcommand{\sumi}{\sum_{i=1}}
\newcommand{\seqi}[1]{\{ #1 \}_{i=1}}
\newcommand{\cupn}{\bigcup_{n=1}}
\newcommand{\capn}{\bigcap_{n=1}}
\newcommand{\sumn}{\sum_{n=1}}
\newcommand{\seqn}[1]{\{ #1 \}_{n=1}}
\newcommand{\cupk}{\bigcup_{k=1}}
\newcommand{\capk}{\bigcap_{k=1}}
\newcommand{\sumk}{\sum_{k=1}}
\newcommand{\seqk}[1]{\{ #1 \}_{k=1}}
\newcommand{\cupm}{\bigcup_{m=1}}
\newcommand{\capm}{\bigcap_{m=1}}
\newcommand{\summ}{\sum_{m=1}}
\newcommand{\seqm}[1]{\{ #1 \}_{m=1}}

% mathfrak 
\newcommand{\afk}{\mathfrak{A}}
\newcommand{\bfk}{\mathfrak{B}}
\newcommand{\hfk}{\mathfrak{H}}

% mathcal 
\newcommand{\acal}{\mathcal{A}}
\newcommand{\ecal}{\mathcal{E}}
\newcommand{\fcal}{\mathcal{F}}
\newcommand{\hcal}{\mathcal{H}}
\newcommand{\ncal}{\mathcal{N}}
\newcommand{\ocal}{\mathcal{O}}
\newcommand{\rcal}{\mathcal{R}}
\def\L{\mathcal{L}}
\def\H{\mathcal{H}}
\renewcommand{\P}{\mathcal{P}}

% Analysis
\newcommand{\om}{\mu^*}
\newcommand{\fs}{F_\sigma}
\newcommand{\gd}{G_\delta}
\newcommand{\mf}{\mathfrak{M}}
\newcommand{\nf}{\mathfrak{N}}
\newcommand{\Sfin}{S_{\text{fin}}}

\DeclareMathOperator{\leb}{Leb}
\DeclareMathOperator{\Rec}{Rec}
\DeclareMathOperator{\Cube}{Cube}
\DeclareMathOperator{\RCube}{RCube}
\newcommand{\RCuber}{\RCube_r}
\DeclareMathOperator{\Reco}{Rec^{\circ}}
\DeclareMathOperator{\Cubeo}{Cube^{\circ}}
\DeclareMathOperator{\RCubeo}{RCube^{\circ}}
\newcommand{\RCubeor}{\RCube^{\circ}_r}

% Matrix groups
\newcommand{\GL}{\mathrm{GL}}
\newcommand{\Or}{\mathrm{O}}
\newcommand{\PGL}{\mathrm{PGL}}
\newcommand{\PSL}{\mathrm{PSL}}
\newcommand{\PSO}{\mathrm{PSO}}
\newcommand{\PSU}{\mathrm{PSU}}
\newcommand{\SL}{\mathrm{SL}}
\newcommand{\SO}{\mathrm{SO}}
\newcommand{\Spin}{\mathrm{Spin}}
\newcommand{\Sp}{\mathrm{Sp}}
\newcommand{\SU}{\mathrm{SU}}
\newcommand{\U}{\mathrm{U}}
\newcommand{\Mat}{\mathrm{Mat}}

% Matrix algebras
\newcommand{\gl}{\mathfrak{gl}}
\newcommand{\ort}{\mathfrak{o}}
\newcommand{\so}{\mathfrak{so}}
\newcommand{\su}{\mathfrak{su}}
\newcommand{\uu}{\mathfrak{u}}
\renewcommand{\sl}{\mathfrak{sl}}

% Special sets
\newcommand{\C}{\mathbb{C}}
\newcommand{\CP}{\mathbb{CP}}
\newcommand{\GG}{\mathbb{G}}
\newcommand{\N}{\mathbb{N}}
\newcommand{\Q}{\mathbb{Q}}
\newcommand{\R}{\mathbb{R}}
\newcommand{\RP}{\mathbb{RP}}
\newcommand{\T}{\mathbb{T}}
\newcommand{\Z}{\mathbb{Z}}
% \renewcommand{\H}{\mathbb{H}}

% Brackets
\renewcommand{\bar}[1]{\overline{#1}}
\renewcommand{\tilde}[1]{\widetilde{#1}}
\renewcommand{\hat}[1]{\widehat{#1}}
\newcommand{\floor}[1]{\left\lfloor #1 \right\rfloor}
\newcommand{\ceil}[1]{\left\lceil #1 \right\rceil}
\newcommand{\abs}[1]{\left\lvert #1\right\rvert}
\newcommand{\bket}[1]{\left\lvert #1\right\rangle}
\newcommand{\brak}[1]{\left\langle #1 \right\rvert}
\newcommand{\braket}[1]{\left\langle #1 \right\rangle}
\newcommand{\bra}{\left\langle}
\newcommand{\ket}{\right\rangle}
\newcommand{\norm}[1]{\left\lVert #1\right\rVert}
\newcommand{\normalorder}[1]{\mathop{:}\nolimits\!#1\!\mathop{:}\nolimits}
\newcommand{\tv}[1]{|#1|}
\newcommand{\bbracket}[1]{\left\llbracket #1 \right\rrbracket}
\renewcommand{\vec}[1]{\boldsymbol{\mathbf{#1}}}
\DeclareMathOperator{\curl}{curl}
\DeclareMathOperator{\diverge}{div}
\DeclareMathOperator{\dist}{dist}

% not-math
\newcommand{\bolds}[1]{{\bfseries #1}}
\newcommand{\cat}[1]{\mathsf{#1}}
\newcommand{\ph}{\,\cdot\,}
\newcommand{\term}[1]{\emph{#1}\index{#1}}
\newcommand{\phantomeq}{\hphantom{{}={}}}
% Probability
\DeclareMathOperator{\Bernoulli}{Bernoulli}
\DeclareMathOperator{\betaD}{beta}
\DeclareMathOperator{\bias}{bias}
\DeclareMathOperator{\binomial}{binomial}
\DeclareMathOperator{\corr}{corr}
\DeclareMathOperator{\cov}{cov}
\DeclareMathOperator{\gammaD}{gamma}
\DeclareMathOperator{\mse}{mse}
\DeclareMathOperator{\multinomial}{multinomial}
\DeclareMathOperator{\Poisson}{Poisson}
\DeclareMathOperator{\var}{var}
\newcommand{\E}{\mathbb{E}}
\renewcommand{\Pr}{\mathbb{P}}

% Algebra
\DeclareMathOperator{\adj}{adj}
\DeclareMathOperator{\Ann}{Ann}
\DeclareMathOperator{\Aut}{Aut}
\DeclareMathOperator{\Char}{char}
\DeclareMathOperator{\disc}{disc}
\DeclareMathOperator{\dom}{dom}
\DeclareMathOperator{\fix}{fix}
\DeclareMathOperator{\Hom}{Hom}
\DeclareMathOperator{\id}{id}
\DeclareMathOperator{\image}{image}
\DeclareMathOperator{\im}{im}
\DeclareMathOperator{\tr}{tr}
\DeclareMathOperator{\Tr}{Tr}
\newcommand{\Bilin}{\mathrm{Bilin}}
\newcommand{\Frob}{\mathrm{Frob}}

% Others
\newcommand\ad{\mathrm{ad}}
\newcommand\Art{\mathrm{Art}}
\newcommand{\B}{\mathcal{B}}
\newcommand{\cU}{\mathcal{U}}
\newcommand{\Der}{\mathrm{Der}}
\newcommand{\D}{\mathrm{D}}
\newcommand{\dR}{\mathrm{dR}}
\newcommand{\exterior}{\mathchoice{{\textstyle\bigwedge}}{{\bigwedge}}{{\textstyle\wedge}}{{\scriptstyle\wedge}}}
\newcommand{\F}{\mathbb{F}}
\newcommand{\G}{\mathcal{G}}
\newcommand{\Gr}{\mathrm{Gr}}
\newcommand{\haut}{\mathrm{ht}}
\newcommand{\Hol}{\mathrm{Hol}}
\newcommand{\hol}{\mathfrak{hol}}
\newcommand{\Id}{\mathrm{Id}}
\newcommand{\lie}[1]{\mathfrak{#1}}
\newcommand{\op}{\mathrm{op}}
\newcommand{\Oc}{\mathcal{O}}
\newcommand{\pr}{\mathrm{pr}}
\newcommand{\Ps}{\mathcal{P}}
\newcommand{\pt}{\mathrm{pt}}
\newcommand{\qeq}{\mathrel{``{=}"}}
\newcommand{\Rs}{\mathcal{R}}
\newcommand{\Vect}{\mathrm{Vect}}
\newcommand{\wsto}{\stackrel{\mathrm{w}^*}{\to}}
\newcommand{\wt}{\mathrm{wt}}
\newcommand{\wto}{\stackrel{\mathrm{w}}{\to}}
% \renewcommand{\d}{\; \mathrm{d}}
\renewcommand{\d}{\, d}
\newcommand{\dd}{\partial}
\newcommand{\nsubset}{\not\subset}
\renewcommand{\epsilon}{\varepsilon}
\renewcommand{\phi}{\varphi}


\let\Im\relax
\let\Re\relax

\DeclareMathOperator{\RHS}{RHS}
\DeclareMathOperator{\LHS}{LHS}
\DeclareMathOperator{\step}{Step}
\DeclareMathOperator{\reg}{Reg}
\DeclareMathOperator{\ran}{range}
\DeclareMathOperator{\area}{area}
\DeclareMathOperator{\card}{card}
\DeclareMathOperator{\ccl}{ccl}
\DeclareMathOperator{\ch}{ch}
\DeclareMathOperator{\cl}{cl}
\DeclareMathOperator{\cls}{\overline{\mathrm{span}}}
\DeclareMathOperator{\coker}{coker}
\DeclareMathOperator{\conv}{conv}
\DeclareMathOperator{\cosec}{cosec}
\DeclareMathOperator{\cosech}{cosech}
\DeclareMathOperator{\covol}{covol}
\DeclareMathOperator{\diag}{diag}
\DeclareMathOperator{\diam}{diam}
\DeclareMathOperator{\Diff}{Diff}
\DeclareMathOperator{\End}{End}
\DeclareMathOperator{\energy}{energy}
\DeclareMathOperator{\erfc}{erfc}
\DeclareMathOperator{\erf}{erf}
\DeclareMathOperator*{\esssup}{ess\,sup}
\DeclareMathOperator{\ev}{ev}
\DeclareMathOperator{\Ext}{Ext}
\DeclareMathOperator{\fst}{fst}
\DeclareMathOperator{\Fit}{Fit}
\DeclareMathOperator{\Frac}{Frac}
\DeclareMathOperator{\Gal}{Gal}
\DeclareMathOperator{\gr}{gr}
\DeclareMathOperator{\hcf}{hcf}
\DeclareMathOperator{\Im}{Im}
\DeclareMathOperator{\Ind}{Ind}
\DeclareMathOperator{\Int}{Int}
\DeclareMathOperator{\Isom}{Isom}
\DeclareMathOperator{\lcm}{lcm}
\DeclareMathOperator{\length}{length}
\DeclareMathOperator{\Lie}{Lie}
\DeclareMathOperator{\like}{like}
\DeclareMathOperator{\Lk}{Lk}
\DeclareMathOperator{\Maps}{Maps}
\DeclareMathOperator{\orb}{orb}
\DeclareMathOperator{\ord}{ord}
\DeclareMathOperator{\otp}{otp}
\DeclareMathOperator{\poly}{poly}
\DeclareMathOperator{\rank}{rank}
\DeclareMathOperator{\rel}{rel}
\DeclareMathOperator{\Rad}{Rad}
\DeclareMathOperator{\Re}{Re}
\DeclareMathOperator*{\res}{res}
\DeclareMathOperator{\Res}{Res}
\DeclareMathOperator{\Ric}{Ric}
\DeclareMathOperator{\rk}{rk}
\DeclareMathOperator{\Rees}{Rees}
\DeclareMathOperator{\Root}{Root}
\DeclareMathOperator{\sech}{sech}
\DeclareMathOperator{\sgn}{sgn}
\DeclareMathOperator{\snd}{snd}
\DeclareMathOperator{\Spec}{Spec}
\DeclareMathOperator{\spn}{span}
\DeclareMathOperator{\stab}{stab}
\DeclareMathOperator{\St}{St}
\DeclareMathOperator{\supp}{supp}
\DeclareMathOperator{\spt}{spt}
\DeclareMathOperator{\Syl}{Syl}
\DeclareMathOperator{\Sym}{Sym}
\DeclareMathOperator{\vol}{vol}

\pgfarrowsdeclarecombine{twolatex'}{twolatex'}{latex'}{latex'}{latex'}{latex'}
\tikzset{->/.style = {decoration={markings,
                                  mark=at position 1 with {\arrow[scale=2]{latex'}}},
                      postaction={decorate}}}
\tikzset{<-/.style = {decoration={markings,
                                  mark=at position 0 with {\arrowreversed[scale=2]{latex'}}},
                      postaction={decorate}}}
\tikzset{<->/.style = {decoration={markings,
                                   mark=at position 0 with {\arrowreversed[scale=2]{latex'}},
                                   mark=at position 1 with {\arrow[scale=2]{latex'}}},
                       postaction={decorate}}}
\tikzset{->-/.style = {decoration={markings,
                                   mark=at position #1 with {\arrow[scale=2]{latex'}}},
                       postaction={decorate}}}
\tikzset{-<-/.style = {decoration={markings,
                                   mark=at position #1 with {\arrowreversed[scale=2]{latex'}}},
                       postaction={decorate}}}
\tikzset{->>/.style = {decoration={markings,
                                  mark=at position 1 with {\arrow[scale=2]{latex'}}},
                      postaction={decorate}}}
\tikzset{<<-/.style = {decoration={markings,
                                  mark=at position 0 with {\arrowreversed[scale=2]{twolatex'}}},
                      postaction={decorate}}}
\tikzset{<<->>/.style = {decoration={markings,
                                   mark=at position 0 with {\arrowreversed[scale=2]{twolatex'}},
                                   mark=at position 1 with {\arrow[scale=2]{twolatex'}}},
                       postaction={decorate}}}
\tikzset{->>-/.style = {decoration={markings,
                                   mark=at position #1 with {\arrow[scale=2]{twolatex'}}},
                       postaction={decorate}}}
\tikzset{-<<-/.style = {decoration={markings,
                                   mark=at position #1 with {\arrowreversed[scale=2]{twolatex'}}},
                       postaction={decorate}}}

\tikzset{circ/.style = {fill, circle, inner sep = 0, minimum size = 3}}
\tikzset{scirc/.style = {fill, circle, inner sep = 0, minimum size = 1.5}}
\tikzset{mstate/.style={circle, draw, blue, text=black, minimum width=0.7cm}}

\tikzset{eqpic/.style={baseline={([yshift=-.5ex]current bounding box.center)}}}
\tikzset{commutative diagrams/.cd,cdmap/.style={/tikz/column 1/.append style={anchor=base east},/tikz/column 2/.append style={anchor=base west},row sep=tiny}}

\definecolor{mblue}{rgb}{0.2, 0.3, 0.8}
\definecolor{morange}{rgb}{1, 0.5, 0}
\definecolor{mgreen}{rgb}{0.1, 0.4, 0.2}
\definecolor{mred}{rgb}{0.5, 0, 0}

\def\drawcirculararc(#1,#2)(#3,#4)(#5,#6){%
    \pgfmathsetmacro\cA{(#1*#1+#2*#2-#3*#3-#4*#4)/2}%
    \pgfmathsetmacro\cB{(#1*#1+#2*#2-#5*#5-#6*#6)/2}%
    \pgfmathsetmacro\cy{(\cB*(#1-#3)-\cA*(#1-#5))/%
                        ((#2-#6)*(#1-#3)-(#2-#4)*(#1-#5))}%
    \pgfmathsetmacro\cx{(\cA-\cy*(#2-#4))/(#1-#3)}%
    \pgfmathsetmacro\cr{sqrt((#1-\cx)*(#1-\cx)+(#2-\cy)*(#2-\cy))}%
    \pgfmathsetmacro\cA{atan2(#2-\cy,#1-\cx)}%
    \pgfmathsetmacro\cB{atan2(#6-\cy,#5-\cx)}%
    \pgfmathparse{\cB<\cA}%
    \ifnum\pgfmathresult=1
        \pgfmathsetmacro\cB{\cB+360}%
    \fi
    \draw (#1,#2) arc (\cA:\cB:\cr);%
}
\newcommand\getCoord[3]{\newdimen{#1}\newdimen{#2}\pgfextractx{#1}{\pgfpointanchor{#3}{center}}\pgfextracty{#2}{\pgfpointanchor{#3}{center}}}

\newcommand\qedshift{\vspace{-17pt}}
\newcommand\fakeqed{\pushQED{\qed}\qedhere}

\def\Xint#1{\mathchoice
   {\XXint\displaystyle\textstyle{#1}}%
   {\XXint\textstyle\scriptstyle{#1}}%
   {\XXint\scriptstyle\scriptscriptstyle{#1}}%
   {\XXint\scriptscriptstyle\scriptscriptstyle{#1}}%
   \!\int}
\def\XXint#1#2#3{{\setbox0=\hbox{$#1{#2#3}{\int}$}
     \vcenter{\hbox{$#2#3$}}\kern-.5\wd0}}
\def\ddashint{\Xint=}
\def\dashint{\Xint-}

\newcommand\separator{{\centering\rule{2cm}{0.2pt}\vspace{2pt}\par}}

\newenvironment{own}{\color{gray!70!black}}{}

\newcommand\makecenter[1]{\raisebox{-0.5\height}{#1}}

\mathchardef\mdash="2D

\newenvironment{significant}{\begin{center}\begin{minipage}{0.9\textwidth}\centering\em}{\end{minipage}\end{center}}
\DeclareRobustCommand{\rvdots}{%
  \vbox{
    \baselineskip4\p@\lineskiplimit\z@
    \kern-\p@
    \hbox{.}\hbox{.}\hbox{.}
  }}
\DeclareRobustCommand\tph[3]{{\texorpdfstring{#1}{#2}}}
\makeatother


\newcommand{\TODO}{\textcolor{red}{\textbf{*** TO-DO ***}}}

\begin{document}
\maketitle

\tableofcontents

\setcounter{section}{2}
\section{Signed measure and differentiation}

\subsection{Signed measures}

\begin{ex}[Folland 3.2]
If $\nu$ a signed measure, $E$ is $\nu$-null if and only if 
$\abs{\nu}(E) = 0$. Also if $\nu$ and $\mu$ are signed measures,
then $\nu \perp \mu$ if and only if $\abs{\nu} \perp \mu$ 
if and only if $\nu^+ \perp \mu$ and $\nu^- \perp \mu$.
\end{ex}

\begin{proof}
It is clear that $\abs{\nu}(E) = 0$ implies that $E$ is $\nu$-null.
Now suppose $E$ is $\nu$-null and let $X = P \cup N$ be the 
Hahn decomposition of $\nu$. Suppose for contradiction 
that $\abs{\nu}(E) > 0$, then it follows that 
$\nu^+(E) > 0$ and $\nu^- (E) > 0$. We then have 
$\nu^+(E \cap P) = \nu^+ (E \cap P) + \nu^+ (E \cap N) = \nu^+(E)
> 0$, but $\nu^- (E \cap P) \leq \nu^- (P) = 0$. Therefore, 
$\nu(E \cap P) > 0$, a contradiction with $E$ being $\nu$-null.
Therefore, $\abs{\nu}(E) = 0$.

Suppose $\nu \perp \mu$, then there is $X = E \cup F$ such that 
$E$ is $\nu$-null and $F$ is $\mu$-null. It follows that 
$\abs{\nu}(E) = 0$, so $\abs{\nu} \perp \mu$. Therefore, 
$\nu \perp \mu$ implies $\abs{\nu} \perp \mu$. It is clear 
that $\abs{\nu} \perp \mu$ implies $\nu^+ \perp \mu$
and $\nu^- \perp \mu$. Now suppose 
$\nu^+ \perp \mu$ and $\nu^- \perp \mu$. Then we have 
$X = E^+ \cup F^+ = E^- \cup F^-$ where $F^+$ is $\nu^+$ null,
$F^-$ is $\nu^-$ null, and $E^\pm$ is $\mu$-null. Let 
$E = E^+ \cup E^-$ and $F = F^+ \cap F^- = E^c$. Then we can 
verify that $E$ is $\mu$-null and $F$ is $\nu$-null. Therefore, 
$\nu \perp \mu$ and the proof is complete.
\end{proof}

\begin{ex}[Folland 3.3]
Let $\nu$ be a signed measure on $(X, \mf)$. 
\begin{enumerate}
  \item $L^1 (\nu) = L^1(\abs{\nu})$.
  \item If $f \in L^1 (\nu)$, then $\abs{\int f \d \nu}
  \leq \int \abs{f} \d \abs{\nu}$.
  \item If $E \in \mf$, $\abs{\nu}(E) = \sup 
  \left\{ \abs{\int_E f \d \nu} : \abs{f} \leq 1 \right\}$
\end{enumerate}  
\end{ex}

\begin{proof}
\begin{enumerate}
\item Since we have 
\[
\int \abs{f} \d \nu = \int \abs{f} \d \nu^+ - \int \abs{f} \d \nu^-, 
\quad 
\int \abs{f} \d \abs{\nu} = \int \abs{f} \d \nu^+ + \int \abs{f} 
\d \nu^-,
\]
it follows immediately that $L^1 (\nu) = L^1 (\abs{\nu})$.

\item We have 
\[
\begin{aligned}
\abs{\int f \d \nu} = \abs{\int f d \nu^+ - \int f \d \nu^-} 
\leq \int \abs{f} \d \nu^+ + \int f \d \nu^- 
= \int \abs{f} \d \abs{\nu}.
\end{aligned}
\]

\item By the previous item, we know
\[
\abs{\nu} (E) = \int_E 1 \d \abs{\nu} \geq \int_E \abs{f} 
\d \abs{\nu} \geq \abs{\int f \d \nu}.
\]
for any $\abs{f} \leq 1$. Also, $\abs{\nu}(E) = 
\int \abs{\chi_P + \chi_N} \d \nu$, where $X = P \cup N$ is the 
Hahn decomposition so $\abs{\chi_P + \chi_N} = 1$.
\end{enumerate}
\end{proof}

\begin{ex}[Folland 3.4]
If $\nu$ is a signed measure and $\lambda, \mu$ are positive
measures such that $\nu = \lambda - \mu$, then $\lambda \geq \nu^+$
and $\mu \geq \nu^-$.
\end{ex}

\begin{proof}
Let $\nu = \nu^+ - \nu^-$ be the unique decomposition. 
We then have $\lambda - \nu^+ = \mu - \nu^-$. Pick a set $E \in \mf$
such that $\nu^- (E) = 0$. We then have 
\[
\lambda (E) - \nu^+ (E) = \mu (E) \geq 0.
\]
Therefore $\lambda (E) \geq \nu^+ (E)$. On the other hand,
for $E \in \mf$ such that $\nu^+ (E) = 0$, we have 
$\lambda (E) \geq 0 = \nu^+ (E)$. In light of the Hahn 
decomposition for $\nu$ and the additivity of measure, 
we can conclude that $\lambda \geq \nu^+$, and thus
$\mu \geq \nu^-$.
\end{proof}

\begin{ex}[Folland 3.5]
If $\nu_1, \nu_2$ are both signed measures that omits the values 
$\pm \infty$, then $\abs{\nu_1 + \nu_2} \leq \abs{\nu_1} + \abs{\nu_2}$.
(Use Exercise 3.4)
\end{ex}

\begin{proof}
Note that we have $\nu_1 + \nu_2 = (\nu_1^+ +  \nu_2^+)
- (\nu_1^- + \nu_2^-)$. Write $\nu = \nu_1 + \nu_2$. 
By Exercise 3.4, we have 
$\nu_1^+ + \nu_2^+ \geq \nu^+$ and $\nu_1^- + \nu_2^- 
\geq \nu^-$. It follows that 
\[
\abs{\nu_1 + \nu_2} = \nu^+ + \nu^- 
\leq \nu_1^+ + \nu_2^+ + \nu_1^- + \nu_2^- 
= \abs{\nu_1} + \abs{\nu_2},
\]
as desired.
\end{proof}

\begin{ex}[Folland 3.6]
Suppose $\nu(E) = \int_E f \d \mu$ where $\mu$ is a positive 
measure and $f$ is an extended $\mu$-integrable function. 
Describe the Hahn decomposition for $\nu$ and express 
the positive, negative, and total variation of $\nu$ 
in terms of $f$ and $\mu$. 
\end{ex}

\begin{proof}
A Hahn decomposition of $\nu$ is $X = P \cup N$ where 
$P = \{f \geq 0\}$ and $N = P^c$. We also have 
\[
\begin{aligned}
  \nu^+ (E) = \int_E f^+ \d \mu, \quad
  \nu^- (E) = \int_E f^- \d \mu, \quad
  \abs{\nu} (E) = \int_E \abs{f} \d \mu.
\end{aligned}
\]
\end{proof}

\begin{ex}[Folland 3.7]
Suppose $\nu$ is a signed measure on $(X, \mf)$ and $E \in \mf$.
\begin{enumerate}
\item $\nu^+ (E) = \sup \left\{ \nu(F) : F \subset E, F \in \mf 
\right\}$ and $\nu^-(E) = - \inf \left\{ \nu(F) : F \subset E, 
F \in \mf \right\}$. 

\item $\abs{\nu}(E) = \sup \left\{ \sumj^n \abs{\nu(E_j)} 
: \text{$n \in \N$, $E_1, \dots, E_n$ disjoint, and 
$\cupj^n E_j = E$} \right\}$.
\end{enumerate}
\end{ex}

\begin{proof}
\begin{enumerate}
\item First let $F \subset E$, we have $\nu(F) = \nu^+(F) - \nu^-(F)$.
It follows that $\nu(F) \leq \nu^+ (F) \leq \nu^+ (E)$. Let 
$X = P \cup N$ be the Hahn decomposition for $\nu$, then we have 
$\nu^+(E) = \nu(E \cap P)$. Similarly for $\nu^- (E)$. 

\item Let $E_1, \dots, E_n$ be disjoint and $\cupj^n E_j = E$. 
It follows that 
\[
\abs{\nu}(E) = \sumj^n \abs{\nu}(E_j) \geq \sumj^n \abs{\nu(E_j)}.
\]
Let $X = P \cup N$ be the Hahn decomposition for $\nu$, we 
then have 
\[
\abs{\nu}(E) = \nu^+ (E) + \nu^-(E) = \nu^+ (E \cap P) 
+ \nu^- (E \cap N) = \abs{\nu (E \cap P)} + \abs{\nu (E \cap N)}.
\]
This completes the proof.
\end{enumerate}
\end{proof}

\subsection{The Lebesgue-Randon-Nikodym theorem}

\begin{ex}[Folland 3.8]
$\nu \ll \mu$ if and only if $\abs{\nu} \ll \mu$ 
if and only if $\nu^+ \ll \mu$ and $\nu^- \ll \mu$.
\end{ex}

\begin{proof}
It is clear that $\abs{\nu} \ll \mu$ implies 
$\nu^+ \ll \mu$ and $\nu^- \ll \mu$ implies 
$\nu \ll \mu$. It then remains to show that $\nu \ll \mu$ 
implies $\abs{\nu} \ll \mu$. Let $X = P \cup N$ be the Hahn 
decomposition for $\nu$ and $E \in \mf$ be such that 
$\mu(E) = 0$. It follows that 
$\mu(E \cap P) = 0$ and $\mu (E \cap N) = 0$. Then 
$\nu(E \cap P) = \nu^+ (E \cap P) = 0$ and 
$\nu(E \cap N) = \nu^- (E \cap N) = 0$. Therefore, 
\[
\nu (E) = \nu^+ (E \cap P) + \nu^- (E \cap N) = 0,
\]
as desired.
\end{proof}

\begin{ex}[Folland 3.9]
Suppose $\{\nu_j\}$ is a sequence of positive measures.
If $\nu_j \perp \mu$ for all $j$, then $\suminfj \nu_j \perp \mu$, 
and if $\nu_j \ll \mu$ for all $j$, then $\suminfj \nu_j 
\perp \mu$.
\end{ex}

\begin{proof}
\TODO
\end{proof}

\subsection{Complex measures}

\begin{ex}[Folland 3.18]
Let $\nu$ be a complex measure on $(X, \mf)$. Prove that 
$L^1 (\nu) = L^1 (\abs{\nu})$ and if $f \in L^1 (\nu)$, then 
$\abs{\int f \d \nu} \leq \int \abs{f} \d \abs{\nu}$.
\end{ex}

\begin{proof}
For $L^1 (\nu) \subset L^1 (\abs{\nu})$, consider $f \in L^1 (\nu)$.
Note that $\nu = \nu_r + i \nu_i$ and it is easy to 
verify that $\abs{i \nu_i} = \abs{\nu_i}$.
Therefore by Proposition 3.14, we have $\abs{\nu} \leq 
\abs{\nu_r} + \abs{\nu_i}$. It follows that 
\[
\begin{aligned}
\int \abs{f} \d \abs{\nu} 
&\leq \int \abs{f} \d \abs{\nu_r} + \int \abs{f} \d
\abs{\nu_i} \\
&= \int \abs{f} \d \nu_r^+ + \int \abs{f} \d \nu_r^- 
+ \int \abs{f} \d \nu_i^+ + \int \abs{f} \d \nu_i^-.
\end{aligned}
\]
Since $f \in L^1 (\nu)$, all four terms are finite and thus 
$f \in L^1 (\abs{\nu})$. 

For $L^1 (\abs{\nu}) \subset L^1 (\nu)$, consider $f \in L^1 
(\abs{\nu})$. Then we have 
\[
\int \abs{f} \d \nu 
= \int \abs{f} \frac{d \nu}{d \abs{\nu}} \d \abs{\nu} 
\leq \int \abs{f} \abs{\frac{d \nu}{d \abs{\nu}}} \d \abs{\nu} 
= \int \abs{f} \d \abs{\nu},
\]
where we have used the fact that $d \nu / d \abs{\nu}$ has 
absolute value $1$ $\abs{\nu}$-a.e. This shows that 
$f \in L^1 (\nu)$. 

Moreover, we have that $\abs{\int f \d \nu} 
\leq \int \abs{f} \d \nu$. Therefore,
\[
\abs{\int f \d \nu} \leq \int \abs{f} \d \abs{\nu},
\]
as desired.
\end{proof}

\begin{ex}[Folland 3.19]
If $\nu, \mu$ are complex measures and $\lambda$ 
is a positive measure, then $\nu \perp \mu$ if and only 
if $\abs{\nu} \perp \abs{\mu}$, and $\nu \ll \lambda$ 
if and only if $\abs{\nu} \ll \lambda$.
\end{ex}

\begin{proof}
The ``if'' direction is clear for both propositions.

Suppose $\nu \perp \mu$. Then $\nu_r \perp \mu_r$,
$\nu_i \perp \mu_r$, $\nu_r \perp \mu_i$, and $\nu_i \perp \mu_i$.
It follows that $\abs{\nu_r} \perp \mu_r$ and $\abs{\nu_i}
\perp \mu_r$. Since $\abs{\nu} \leq \abs{\nu_r} + \abs{\nu_i}$, 
we have $\abs{\nu} \perp \mu_r$. Similarly $\abs{\nu} \perp \mu_i$.
Following the same reasoning, we obtain 
$\abs{\nu} \perp \abs{\mu}$, as desired.

Suppose $\nu \ll \lambda$. 
Since $\nu = \nu_r + i \nu_i$, we have 
$\nu_r \ll \lambda$ and $\nu_i \ll \lambda$. 
Recall from Exercise 3.8 that this implies $\abs{\nu_r} \ll 
\lambda$ and $\abs{\nu_i} \ll \lambda$. Moreover, 
$\abs{\nu} \leq \abs{\nu_r} + \abs{\nu_i}$, so 
$\abs{\nu} \ll \lambda$.
\end{proof}

\begin{ex}[Folland 3.20]
If $\nu$ is a complex measure on $(X, \mf)$ and 
$\nu(X) = \abs{\nu}(X)$, then $\nu = \abs{\nu}$.
\end{ex}

\begin{proof}
By Lebesgue-Randon-Nikodym theorem, we have 
$d \nu = f \d \mu$ for some function $f$ and positive 
measure $\mu$. It follows that $d \abs{\nu} 
= \abs{f} \d \mu$ and 
\[
\int f \d \mu = \int \abs{f} \d \mu.
\]
Now let $E \in \mf$. We then have 
\[
\int_E f \d \mu + \int_{E^c} f \d \mu = 
\int_E \abs{f} \d \mu + \int_{E^c} \abs{f} \d \mu.
\]
It follows that 
\[
0 \leq \int_{E} \abs{f} - f_r \d \mu
= \int_{E^c} f_r - \abs{f} \d \mu \leq 0.
\]
Since $E \in \mf$ is arbitrary, 
$f_r = \abs{f}$ and $f_i = 0$ a.e. It follows that 
$d \nu = f \d \mu = \abs{f} \d \mu = d \abs{\nu}$.
\end{proof}

\begin{ex}[Folland 3.21]
Let $\nu$ be a complex measure on $(X, \mf)$. If $E \in \mf$, 
define 
\[
\begin{aligned}
\mu_1 (E) 
&= \sup \left\{ \sumj^n \abs{\nu(E_j)} : \text{$n \in \N$, 
$E_1, \dots, E_n$ disjoint, $E = \cupj^n E_j$} \right\}, \\
\mu_2 (E) 
&= \sup \left\{ \suminfj \abs{\nu(E_j)} : 
\text{$E_1, E_2, \dots$ disjoint, $E = \cupinfj E_j$} \right\}, \\
\mu_3 (E) 
&= \sup \left\{ \abs{\int_E f \d \nu} : \abs{f} \leq 1 \right\}.
\end{aligned}
\]
Then $\mu_1 = \mu_2 = \mu_3 = \abs{\nu}$. (First show that 
$\mu_1 \leq \mu_2 \leq \mu_3$. To see $\mu_3 = \abs{\nu}$, 
let $f = \bar{d \nu / d \abs{\nu}}$ and apply Proposition 3.13. 
To see $\mu_3 \leq \mu_1$, approximate $f$ by simple functions.)
\end{ex}

\begin{proof}
It is clear that $\mu_1 \leq \mu_2$ by letting $E_j = \emptyset$
for $j > n$. To show $\mu_2 \leq \mu_3$, 
consider disjoint collection of sets $\seqinfj{E_j}$ such 
that $E = \cupinfj E_j$. Let 
\[
f = \sumj^\infty \frac{\bar{\nu(E_j)}}{\abs{\nu(E_j)}} \chi_{E_j}.
\]
Since $\seqinfj{E_j}$ are disjoint, we have $\abs{f} \leq 1$.
Moreover, $1$ is integrable on $X$, 
so dominated convergence theorem implies 
\[
\int_E f \d \nu 
= \int_E \sumj^\infty \frac{\bar{\nu(E_j)}}{\abs{\nu(E_j)}} 
\chi_{E_j} \d \nu
= \sumj^\infty \int_E \frac{\bar{\nu(E_j)}}{\abs{\nu(E_j)}} 
\chi_{E_j} \d \nu
= \suminfj \abs{\nu(E_j)}.
\]
Therefore, $\mu_2 \leq \mu_3$.

Now we show $\mu_3 = \abs{\nu}$. 
It is clear that 
\[
\abs{\int_E f \d \nu} \leq \int_E \abs{f} \d \abs{\nu}
\leq \abs{\nu} (E),
\]
so $\mu_3 \leq \abs{\nu}$.
On the other hand, let $f = \bar{d \nu / d \abs{\nu}}$.
Then, $\abs{f} = 1$ a.e. and 
\[
\abs{\int_E \bar{\frac{d \nu}{d \abs{\nu}}} \d \nu} 
= \abs{\int_E \bar{\frac{d \nu}{d \abs{\nu}}} 
\frac{d \nu}{d \abs{\nu}} \d \abs{\nu}}
= \int_E 1 \d \abs{\nu}
= \abs{\nu} (E).
\]
Therefore, $\abs{\nu} \leq \mu_3$ and thus $\abs{\nu} 
= \mu_3$.

Finally, to show $\mu_3 \leq \mu_1$, let $\phi$ be a simple 
function such that $\abs{\phi} \leq 1$. Let 
\[
\phi = \sumj^n v_j \chi_{E_j}
\]
be the canonical representation of $\phi$. We then have 
$\abs{v_j} \leq 1$ for each $j$ and $E_1, \dots, E_n$ 
are disjoint. WLOG assume 
$E = \cupj^n E_j$. It follows that 
\[
\begin{aligned}
\abs{\int_E \phi \d \nu}
= \abs{\sumj^n v_j {\nu(E_j)}}
\leq \sumj^n \abs{\nu (E_j)} \leq \mu_1 (E).
\end{aligned}
\]
We know any function $f$ with $\abs{f} \leq 1$ can 
be approximated by simple functions with absolute 
value less than or equal to $1$, so $\mu_3 \leq \mu_1$. 
The proof is then complete.
\end{proof}

\end{document}