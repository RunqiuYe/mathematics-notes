\documentclass[a4paper]{article}
\usepackage{parskip}
\usepackage{lipsum}

\def\nterm {Spring}
\def\nyear {2025}
\def\ncourse {Introduction to Functional Analysis}

\input{../header}

\newcommand{\TODO}{\textcolor{red}{\textbf{*** TO-DO ***}}}

\begin{document}
\maketitle

\tableofcontents

\section{Banach space theory}

\subsection{Quotient spaces, Baire category and 
uniform boundedness}

\begin{thm}
Let $\norm{\cdot}$ be a \textbf{seminorm} on a vector 
space $V$. If we define $E = \left\{ v \in V: \norm{v} 
= 0 \right\}$, then $E$ is a subspace of $V$, 
and the function on $V / E$ defined by 
\[
\norm{v + E} = \norm{v}
\]
for any $v + E \in V / E$ defines a \textbf{norm}.
\end{thm}

\begin{thm}[Baire Category Theorem]
Let $M$ be a complete metric space, and let $\seqinfn{C_n}$
be a collection of closed subsets of $M$ such that 
$M = \cupinfn C_n$. Then at least one of the $C_n$ contains
an open ball $B(x, r) = \left\{ y \in M : d(x, y) < r \right\}$.
\end{thm}

\begin{thm}[Uniform Boundedness Theorem]
Let $B$ be Banach space and $V$ a normed vector space. 
Let $\seqinfn{T_n}$ be a sequence in $\B(B, V)$.
Then if for all $b \in B$ we have $\sup_n \norm{T_n b} < \infty$
(that is, this sequence is pointwise boudned),
then $\sup_n \norm{T_n} < \infty$ (the operator norms are 
bounded).
\end{thm}

\begin{proof}
For each $k \in \N$, define 
\[
C_k = \left\{ b \in B: \norm{b} \leq 1, \sup_{n \in \N} 
\norm{T_n b} \leq k \right\}.
\]
This set is closed for each $k \in \N$, but by assumption, 
we have 
\[
\left\{ b \in B: \norm{b} \leq 1 \right\} = \cupinfk C_k.
\]
The left hand side is a closed subset of $B$, and is thus 
a complete metric space. By Baire Category Theorem, 
there exists $k \in \N$ such that $C_k$ contains an open 
ball $B(b_0, \delta_0)$. Then, if $b \in B(0, \delta_0)$, 
we have $b_0 + b \in B(b_0, \delta_0)$ and thus 
\[
\sup_{n \in \N} \norm{T_n (b_0 + b)} \leq k.
\]
It follows that 
\[
\sup_{n \in \N} \norm{T_n b} \leq \sup_{n \in \N}
\norm{T_n (b_0 + b)} + \sup_{n \in \N} \norm{T_n b_0}
\leq 2k.
\]
Suppose $\norm{b} = 1$, then $\frac{\delta_0}{2} b \in 
B(0, \delta_0)$ and thus for all $n \in \N$, 
we have 
\[
\norm{T_n \left( \frac{\delta_0}{2} b \right)} \leq 2k.
\]
Therefore, 
\[
\sup_{n \in \N} \norm{T_n} \leq \frac{4k}{\delta_0}.
\]
\end{proof}

\section{Hilbert space theory}

\subsection{Basic Hilbert space theory}

\begin{defi}[Pre-Hilbert space]
  A \textbf{pre-Hilbert} space $H$ is a vector space over 
  $\C$ with a \textbf{Hermitian inner product}, which is a 
  map $\braket{\cdot}{\cdot} : H \times H \to \C$ satisfying
  the following properties.

  \begin{enumerate}
    \item For all $\lambda_1, \lambda_2 \in C$ and 
    $v_1, v_2, w \in H$, we have 
    \[
    \braket{\lambda_1 v_1 + \lambda_2 v_2}{w} 
    = \lambda_1 \braket{v_1}{2} + \lambda_2 \braket{v_2}{w}.
    \]

    \item For all $v, w \in H$, we have $\braket{v}{w} 
    = \bar{\braket{w}{v}}$.

    \item For all $v \in H$, we have $\braket{v}{v} \geq 0$, 
    with equality if and only if $v = 0$.
  \end{enumerate}
\end{defi}

\begin{defi}
Let $H$ be a pre-Hilbert space. For all $v \in H$, we 
define 
\[
\norm{v} = \braket{v}{v}^{\frac{1}{2}}.
\]
\end{defi}

\begin{thm}[Cauchy-Schwarz inequality]
Let $H$ be a pre-Hilbert space. For all $u, v \in H$, we
have 
\[
\abs{\braket{u}{v}} \leq \norm{u} \norm{v}.
\]
\end{thm}

\begin{proof}
Define $f(t) = \norm{u + tv}^2$. Notice that 
\[
\begin{aligned}
  f(t) 
  &= \braket{u + tv}{u + tv}  \\
  &= \braket{u}{u} 
  + t^2 \braket{v}{v} + t \braket{u}{v} + t \braket{v}{u} \\
  &= \norm{u}^2 + t^2 \norm{v}^2 + 2t \Re(\braket{u}{v}).
\end{aligned}
\]  
This implies that 
\[
0 \leq f(t_{\text{min}}) = \norm{u}^2 - 
\frac{\Re(\braket{u}{v})^2}{\norm{v}^2}.
\]
It follows that 
\[
\abs{\Re( \braket{u}{v} )} \leq \norm{u} \norm{v}.
\]
This is almost what we want. 
To finish up, first note that if 
$\braket{u}{v} = 0$ then there is nothing to prove,
so suppose $\braket{u}{v} \neq 0$,
and define 
\[
\lambda = \frac{\bar{\braket{u}{v}}}{\abs{\braket{u}{v}}}.
\]
Note that we have $\abs{\lambda} = 1$ and we have 
the chain of equalities of real numbers:
\[
\abs{\braket{u}{v}} = \lambda \braket{u}{v} 
= \braket{\lambda u}{v} = \Re \braket{\lambda u}{v} 
\leq \norm{\lambda u} \norm{v}.
\]
However, $\norm{\lambda u} = \norm{u}$, so the proof is 
complete.

\end{proof}



\end{document}