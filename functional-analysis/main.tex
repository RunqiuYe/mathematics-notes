\documentclass[a4paper]{article}
\usepackage{parskip}
\usepackage{lipsum}

\def\nterm {Spring}
\def\nyear {2025}
\def\ncourse {Introduction to Functional Analysis}

\input{../header}

\newcommand{\TODO}{\textcolor{red}{\textbf{*** TO-DO ***}}}

\begin{document}
\maketitle

\tableofcontents

\section{Banach space theory}

\subsection{Quotient spaces, Baire category and 
uniform boundedness}

\begin{thm}
Let $\norm{\cdot}$ be a \textbf{seminorm} on a vector 
space $V$. If we define $E = \left\{ v \in V: \norm{v} 
= 0 \right\}$, then $E$ is a subspace of $V$, 
and the function on $V / E$ defined by 
\[
\norm{v + E} = \norm{v}
\]
for any $v + E \in V / E$ defines a \textbf{norm}.
\end{thm}

\begin{thm}[Baire Category Theorem]
Let $M$ be a complete metric space, and let $\seqinfn{C_n}$
be a collection of closed subsets of $M$ such that 
$M = \cupinfn C_n$. Then at least one of the $C_n$ contains
an open ball $B(x, r) = \left\{ y \in M : d(x, y) < r \right\}$.
\end{thm}

\begin{thm}[Uniform Boundedness Theorem]
Let $B$ be Banach space and $V$ a normed vector space. 
Let $\seqinfn{T_n}$ be a sequence in $\B(B, V)$.
Then if for all $b \in B$ we have $\sup_n \norm{T_n b} < \infty$
(that is, this sequence is pointwise boudned),
then $\sup_n \norm{T_n} < \infty$ (the operator norms are 
bounded).
\end{thm}

\begin{proof}
For each $k \in \N$, define 
\[
C_k = \left\{ b \in B: \norm{b} \leq 1, \sup_{n \in \N} 
\norm{T_n b} \leq k \right\}.
\]
This set is closed for each $k \in \N$, but by assumption, 
we have 
\[
\left\{ b \in B: \norm{b} \leq 1 \right\} = \cupinfk C_k.
\]
The left hand side is a closed subset of $B$, and is thus 
a complete metric space. By Baire Category Theorem, 
there exists $k \in \N$ such that $C_k$ contains an open 
ball $B(b_0, \delta_0)$. Then, if $b \in B(0, \delta_0)$, 
we have $b_0 + b \in B(b_0, \delta_0)$ and thus 
\[
\sup_{n \in \N} \norm{T_n (b_0 + b)} \leq k.
\]
It follows that 
\[
\sup_{n \in \N} \norm{T_n b} \leq \sup_{n \in \N}
\norm{T_n (b_0 + b)} + \sup_{n \in \N} \norm{T_n b_0}
\leq 2k.
\]
Suppose $\norm{b} = 1$, then $\frac{\delta_0}{2} b \in 
B(0, \delta_0)$ and thus for all $n \in \N$, 
we have 
\[
\norm{T_n \left( \frac{\delta_0}{2} b \right)} \leq 2k.
\]
Therefore, 
\[
\sup_{n \in \N} \norm{T_n} \leq \frac{4k}{\delta_0}.
\]
\end{proof}

\section{Hilbert space theory}

\subsection{Basic Hilbert space theory}

\begin{defi}[Pre-Hilbert space]
  A \textbf{pre-Hilbert} space $H$ is a vector space over 
  $\C$ with a \textbf{Hermitian inner product}, which is a 
  map $\braket{\cdot}{\cdot} : H \times H \to \C$ satisfying
  the following properties.

  \begin{enumerate}
    \item For all $\lambda_1, \lambda_2 \in C$ and 
    $v_1, v_2, w \in H$, we have 
    \[
    \braket{\lambda_1 v_1 + \lambda_2 v_2}{w} 
    = \lambda_1 \braket{v_1}{2} + \lambda_2 \braket{v_2}{w}.
    \]

    \item For all $v, w \in H$, we have $\braket{v}{w} 
    = \bar{\braket{w}{v}}$.

    \item For all $v \in H$, we have $\braket{v}{v} \geq 0$, 
    with equality if and only if $v = 0$.
  \end{enumerate}
\end{defi}

\begin{defi}
Let $H$ be a pre-Hilbert space. For all $v \in H$, we 
define 
\[
\norm{v} = \braket{v}{v}^{\frac{1}{2}}.
\]
\end{defi}

\begin{thm}[Cauchy-Schwarz inequality]
Let $H$ be a pre-Hilbert space. For all $u, v \in H$, we
have 
\[
\abs{\braket{u}{v}} \leq \norm{u} \norm{v}.
\]
\end{thm}

\begin{proof}
Define $f(t) = \norm{u + tv}^2$. Notice that 
\[
\begin{aligned}
  f(t) 
  &= \braket{u + tv}{u + tv}  \\
  &= \braket{u}{u} 
  + t^2 \braket{v}{v} + t \braket{u}{v} + t \braket{v}{u} \\
  &= \norm{u}^2 + t^2 \norm{v}^2 + 2t \Re(\braket{u}{v}).
\end{aligned}
\]  
This implies that 
\[
0 \leq f(t_{\text{min}}) = \norm{u}^2 - 
\frac{\Re(\braket{u}{v})^2}{\norm{v}^2}.
\]
It follows that 
\[
\abs{\Re( \braket{u}{v} )} \leq \norm{u} \norm{v}.
\]
This is almost what we want. 
To finish up, first note that if 
$\braket{u}{v} = 0$ then there is nothing to prove,
so suppose $\braket{u}{v} \neq 0$,
and define 
\[
\lambda = \frac{\bar{\braket{u}{v}}}{\abs{\braket{u}{v}}}.
\]
Note that we have $\abs{\lambda} = 1$ and we have 
the chain of equalities of real numbers:
\[
\abs{\braket{u}{v}} = \lambda \braket{u}{v} 
= \braket{\lambda u}{v} = \Re \braket{\lambda u}{v} 
\leq \norm{\lambda u} \norm{v}.
\]
However, $\norm{\lambda u} = \norm{u}$, so the proof is 
complete.

\end{proof}

\begin{thm}
If $H$ is a pre-Hilbert space, then $\norm{\cdot}$ 
is a norm on $H$.
\end{thm}

\begin{proof}
  Note that 
  \[
  \norm{v} = 0 \iff \braket{v}{v} = 0 \iff v = 0.
  \]
  Now if $\lambda \in \C$ and $v \in H$, then 
  \[
  \braket{\lambda v}{\lambda v} = \lambda \bar{\lambda}
  \braket{v}{v} = \abs{\lambda}^2 \norm{v}^2.
  \]
  Therefore, $\norm{\lambda v} = \abs{\lambda} \norm{v}$. 
  
  Finally, let $u, v \in H$, then 
  \[
  \begin{aligned}
    \norm{u + v}^2 
    &= \braket{u + v}{u + v} \\
    &= \norm{u}^2 + \norm{v}^2 + 2 \Re \braket{u}{v} \\
    &\leq \norm{u}^2 + \norm{v}^2 + 2 \abs{\braket{u}{v}} \\
    &\leq \norm{u}^2 + \norm{v}^2 + 2 \norm{u} \norm{v} \\
    &= (\norm{u} + \norm{v})^2.
  \end{aligned}
  \]
  This completes the proof.
\end{proof}

\begin{thm}
If $u_n \to u$ and $v_n \to v$ in a pre-Hilbert space $H$, 
then $\braket{u_n}{v_n} \to \braket{u}{v}$.
\end{thm}

\begin{proof}
  If $u_n \to u$ and $v_n \to v$, then 
  $\norm{u_n - u} \to 0$ and $\norm{v_n - v} \to 0$. 
  It follows that 
  \[
  \begin{aligned}
    \abs{\braket{u_n}{v_n} - \braket{u}{v}} 
    &= \abs{\braket{u_n - u}{v_n} - \braket{u}{v - v_n}} \\
    &\leq \abs{\braket{u_n - u}{v_n}} 
    + \abs{\braket{u}{v - v_n}} \\
    &\leq \norm{u_n - u} \norm{v_n} + \norm{u} \norm{v - v_n} \\
    &\leq \norm{u_n - u} \sup_{k \in \N} \norm{v_k} + \norm{u} 
    \norm{v - v_n} \\
    &\to 0 
  \end{aligned}
  \]
  as $n \to \infty$. This completes the proof.
\end{proof}

\begin{defi}[Hilbert space]
A \textbf{Hilbert space} is a pre-Hilbert space that is 
complete with repsect to the norm 
$\norm{\cdot} = \braket{\cdot}{\cdot}^{\frac{1}{2}}$.
\end{defi}

\begin{eg}
Some examples of Hilbert spaces: 
\begin{itemize}
  \item $\C^n = \left\{ z = (z_1, \dots, z_n) : z_j \in \C \right\}$
  with
  $\braket{z}{w} = \sum_j z_j \bar{w_j}$ is a Hilbert 
  space.

  \item $\ell^2 = \left\{ a = \seqinfk{a_k} : 
  \text{$a_k \in \C$, $\suminfk \abs{a_k}^2 < \infty$} \right\}$ 
  with
  $\braket{a}{b} = \suminfk a_k \bar{b_k}$ 
  is a Hilbert space.

  \item If $E \subset \R$ is measurable, then 
  $L^2(E) = \left\{ f: \text{$E \to \C$, 
  $\int_E \abs{f}^2 < \infty$} \right\}$ with 
  $\braket{f}{g} = \int_E f \bar{g}$ is a Hilbert 
  space.
\end{itemize}
We will show that each separable Hilbert spaces is 
isometrically isomorphic to either $\C^n$ or $\ell^2$.
\end{eg}

Now we have seen that $\ell^2$ and $L^2$ spaces are 
Hilbert spaces. This is expected since the definition 
of the inner product in these spaces uses the fact 
that they are $\ell^2$ or $L^2$. 
A natural question then is whether
other $\ell^p$ or $L^p$ spaces are also Hilbert spaces
with respect to some inner product?
It turns out there is a simple way to decide whether a 
norm come from a inner-product, and thus whether a 
Banach space is a Hilbert space. 

\begin{thm}[Parallelogram Law]
If $H$ is a pre-Hilbert space, then for all $u, v \in H$, 
we have 
\[
  \norm{u + v}^2 + \norm{u - v}^2 = 2 \left( \norm{u}^2 
  + \norm{v}^2 \right).
\]
In addition, if $H$ is a normed vector space satisfying this 
equality, then $H$ is a pre-Hilbert space.
\end{thm}

Using the previous theorem, we can verify that 
$\ell^p$ and $L^p$ with $p \neq 2$ are \textbf{not} 
Hilbert spaces.

\begin{defi}[Orthogonal]
  If $H$ is a pre-Hilbert space, $u, v \in H$ are 
  \textbf{orthogonal} if $\braket{u}{v} = 0$. 
  We denote this as $u \perp v$.
\end{defi}

\begin{defi}[Orthonormal sets]
  If $H$ is a pre-Hilbert space, a subset $\left\{ e_\lambda 
  \right\}_{\lambda \in \Lambda} \subset H$ is 
  \textbf{orthonormal} if for all $\lambda \in \Lambda$,
  we have $\norm{e_\lambda} = 1$ and $\lambda_1 \neq \lambda_2$
  implies $e_{\lambda_1} \perp e_{\lambda_2}$.
\end{defi}

\begin{remark}
  we will mainly be interested in the case where we have a 
  countable orthonormal set.
\end{remark}

\begin{eg}
  The set $\left\{ \frac{e^{i n x}}{\sqrt{2\pi}} \right\}_{n \in \Z}$
  as elements in $L^2([-\pi, \pi])$ 
  is an orthonormal subset of $L^2([-\pi, \pi])$. 
  Indeed, for any $m, n \in \Z$, we have 
  \[
  \int_{-\pi}^\pi e^{imx} \bar{e^{inx}} 
  = \int_{-\pi}^\pi e^{i(m - n)x}
  = \begin{cases}
    2 \pi & (m = n), \\
    0 & (m \neq n).
  \end{cases}
  \]
  Therefore, $\braket{\frac{e^{inx}}{\sqrt{2\pi}}}{\frac{e^{imx}}
  {\sqrt{2\pi}}} = \delta_{mn}$, and $\left\{ \frac{e^{i n x}}{\sqrt{2\pi}}
  \right\}_{n \in \Z}$ is an orthonormal subset 
  of $L^2([-\pi, \pi])$.
\end{eg}

\begin{thm}[Bessel]
If $\seqinfn{e_n}$ is countable orthonormal subset 
of a pre-Hilbert space $H$, then for all 
$u \in H$, we have 
\[
\suminfn \abs{\braket{u}{e_n}}^2 \leq \norm{u}^2.
\]
\end{thm}

\begin{proof}
  We first do the finite case. Suppose $\left\{ e_n \right\}
  _{n=1}^N$ is an orthonormal subset of $H$. Then, 
  \[
  \begin{aligned}
  \norm{\sum_{n=0}^N \braket{u}{e_n} e_n}^2 
  &= \braket{\sum_{n=0}^N \braket{u}{e_n} e_n }{\sum_{n=0}^N \braket{u}{e_n} e_n} \\
  &= \sum_{n=0}^N 
  \sum_{m=1}^N \braket{u}{e_n} \bar{\braket{u}{e_m}} \braket{e_n}{e_m} \\
  &= \sum_{n=0}^N \abs{\braket{u}{e_n}}^2.
  \end{aligned}
  \]
  Also, 
  \[
  \begin{aligned}
  \braket{u}{\sum_{n=0}^N \braket{u}{e_n}e_n} 
  &= \sum_{n=0}^N \bar{\braket{u}{e_n}} \braket{u}{e_n} \\
  &= \sum_{n=0}^N \abs{\braket{u}{e_n}}^2.
  \end{aligned}
  \]
  Therefore, 
  \[
  \begin{aligned}
  0 
  &\leq \norm{u - \sum_{n=0}^N \braket{u}{e_n}e_n}^2 \\
  &= \norm{u}^2 + \norm{\sum_{n=0}^N \braket{u}{e_n}e_n}^2 
  - 2 \Re \braket{u}{\sum_{n=0}^N \braket{u}{e_n}e_n} \\
  &= \norm{u}^2 - \sum_{n=0}^N \abs{\braket{u}{e_n}}^2,
  \end{aligned}
  \]
  as desired.

  For the infinite case, just take the limit as  
  $N \to \infty$.
\end{proof}

\begin{defi}[Maximal orthonormal subset]
  An orthonormal subset $\left\{ e_\lambda \right\}_\lambda$
  of a pre-Hilbert space is \textbf{maximal} if $u \in H$ 
  and $\braket{u}{e_\lambda} = 0$ for all $\lambda \in \Lambda$
  implies that $u = 0$.
\end{defi}

\begin{thm}
  Every non-trivial pre-Hilbert space has a maximal 
  orthonormal subset.
\end{thm}

This can be proved using Zorn's Lemma. We will prove 
something less strong but often equally useful by hand, 
without applying Zorn's Lemma.

\begin{thm}
  Every non-trivial separable pre-Hilbert space 
  has a countable maximal orthonormal subset.
\end{thm}

\begin{proof}
Use the Gram-Schimdt process.
Let $\seqinfj{v_j}$ be a countable dense subset of $H$
where $v_0 \neq 0$. Claim that for any $n \in \N$, 
there exists $m(n) \leq n$ and an orthonormal subset 
$\left\{ e_1, \dots, e_{m(n)} \right\}$ such that 
\begin{enumerate}
  \item $\spn \left\{ e_1, \dots, e_{m(n)} \right\} 
   = \spn \left\{ v_1, \dots, v_n \right\}$.
  
  \item 
  If $v_n \in \spn \left\{ v_1, \dots, v_{n-1} 
  \right\}$, we have 
  \[
    \left\{ e_1, \dots, e_{m(n)} \right\}  
    = \left\{ e_1, \dots, e_{m(n-1)} \right\} \cup 
    \emptyset.
  \]
  Otherwise, we have
  \[
    \left\{ e_1, \dots, e_{m(n)} \right\}  
    = \left\{ e_1, \dots, e_{m(n-1)} \right\} \cup 
    e_{m(n)}
  \]
  for some $e_{m(n)} \in H$.
\end{enumerate}
Prove this by induction. For the base case, 
let $e_1 = \frac{v_1}{\norm{v_1}}$. 
For the inductive step, suppose the claim holds for 
$n = k$. If $v_{k+1} \in \spn \left\{ v_1, \dots, v_k \right\}$,
then 
\[
\spn \left\{ e_1, \dots, e_{n(k)} \right\}
= \spn \left\{ v_1, \dots, v_k \right\} 
= \spn \left\{ v_1, \dots, v_{k+1} \right\}.
\]
Now suppose $v_{k+1} \notin \spn \left\{ v_1, \dots, v_k \right\}$.
Define 
\[
w_{k+1} = v_{k+1} - \sum_{j=1}^{m(k)} \braket{v_{k+1}}{e_j} e_j.
\]
Note that $w_{k+1} \neq 0$ and define $e_{m(k+1)} = 
\frac{w_{k+1}}{\norm{w_{k+1}}}$. Then, $\norm{e_{m(k+1)}} = 1$
and for all $1 \leq l \leq m(k)$, 
\[
\begin{aligned}
  \braket{e_{m(k+1)}}{e_l} 
  &= \frac{1}{\norm{w_{k+1}}}
  \braket{v_{k+1} - \sum_{j=1}^{m(k)} \braket{v_{k+1}}{e_j}}{e_l} \\
  &= \frac{1}{\norm{w_{k+1}}} 
  \left( \braket{v_{k+1}}{e_l} - \braket{v_{k+1}}{e_l} \right) \\
  &= 0.
\end{aligned}
\]
Therefore, $e_{m(k+1)}$ is the desired vector we want and 
we have completed the proof for the claim.

Now let 
\[
S = \cupinfn \left\{ e_1, \dots, e_{m(n)} \right\}.
\]
Then $S$ is a countable orthonormal subset of $H$.
Now we show $S$ is maximal. Suppose $u \in H$ and 
$\braket{u}{e_l} = 0$. Since $\seqinfj{v_j}$ is dense in 
$H$, there exists $\seqinfk{v_{j(k)}}$ such that 
$v_{j(k)} \to u$ as $k \to \infty$. 
By our claim, we know 
$v_{j(k)} \in \spn \left\{ e_1, \dots, e_{m(j(k))} \right\}$.
By Bessel's inequality, 
\[
\begin{aligned}
\norm{v_{j(k)}}^2 
= \sum_{l=1}^{m(j(k))} \abs{\braket{v_{j(k)}}{e_l}}^2 
= \sum_{l=1}^{m(j(k))} \abs{\braket{v_{j(k)} - u}{e_l}}^2 
\leq \norm{v_{j(k)} - u}^2,
\end{aligned}
\]
where for the first equality we used the fact that 
$v_{j(k)} \in \spn \left\{ e_1, \dots, e_{m(j(k))} \right\}$.
Since $v_{j(k)} \to u$ as $k \to \infty$, 
the inequality implies that
$\norm{v_{j(k)}} \to 0$ as $k \to \infty$ 
and thus $\norm{u} = 0$, showing that 
$S$ is indeed a maximal orthonormal subset 
of $H$.
\end{proof}

\begin{cor}
$\ell^2$ and $L^2$ have countable maximal orthonormal
subset since they are both separable.
\end{cor}

\subsection{Orthonormal bases and Fourier Series}

\begin{defi}[Orthonormal basis]
Let $H$ be a Hilbert space. An \textbf{orthonormal basis}
of $H$ is a countable maximal orthonormal subset 
$\left\{ e_n \right\}_n$ of $H$.
\end{defi}

\begin{thm}
If $\seqinfn{e_n}$ is an orthonormal basis in 
Hilbert space $H$, then for all $u \in H$, we have 
\[
\suminfn \braket{u}{e_n} e_n = u.
\]
This is the Fourier-Bessel series.

This tells us we can write each element in $H$ as a 
infinite linear combination of the orthonormal basis.
\end{thm}

\begin{proof}
We first prove the sequence of partial sums 
$\left\{ \sum_{n=0}^m \braket{u}{e_n} e_n \right\}_m$
is Cauchy. Let $\epsilon > 0$. By Bessel's inequality, 
we have 
\[
\suminfn \abs{\braket{u}{e_n}}^2 \leq \norm{u}^2 < \infty.
\]
Therefore, there exsits $M \in \N$ such that 
$N \geq M$ implies $\sum_{n=N+1}^\infty \abs{\braket{u}{e_n}}^2 
< \epsilon^2$. Then for all $m > l \geq M$, we have 
\[
\norm{\sum_{n=0}^m \braket{u}{e_n}e_n - 
\sum_{n=0}^l \braket{u}{e_n}e_n}^2 
\leq \sum_{n=l+1}^m \abs{\braket{u}{e_n}}^2 
\leq \sum_{n=l+1}^\infty \abs{\braket{u}{e_n}}^2 
< \epsilon^2.
\]
Therefore, the sequence of partial sum is Cauchy.
Since $H$ is complete, there exists $\bar{u} \in H$
such that $\bar{u} = \suminfn \braket{u}{e_n}e_n$. 
It remains to show that $\bar{u} = u$. 
By continuity of inner-product, for all $l \in \N$, we 
have 
\[
\begin{aligned}
  \braket{u - \bar{u}}{e_l} 
  &= \lim_{m \to \infty} \braket{u - \sum_{n=0}^m 
  \braket{u}{e_n} e_n}{e_l}  \\
  &= \lim_{m \to \infty} \left[ \braket{u}{e_l} 
  - \sum_{n=0}^m \braket{u}{e_n} \braket{e_n}{e_l} \right] \\
  &= 0.
\end{aligned}
\]
Since $\seqinfn{e_n}$ is maximal, this implies that 
$u - \bar{u} = 0$ and the proof is complete.
\end{proof}

\begin{thm}
Let $H$ be a Hilbert space.
If $H$ has an orthonormal basis, then 
$H$ is separable.
\end{thm}

\begin{proof}
Suppose $\seqinfn{e_n}$ is an orthonormal basis for $H$. 
Then 
\[
S = \bigcup_{m \in \N} \left\{ \sum_{n=0}^m q_n e_n : 
q_n \in \Q + i \Q \right\}
\]
is a countable set. 
Also, by the previous theorem, $S$ is dense in $H$.
\end{proof}

\begin{remark}
Let $H$ be a Hilbert space. $H$ is separable if and only if 
$H$ has an orthonormal basis.
\end{remark}

\begin{thm}[Parseval's identity]
If $H$ is a Hilbert space and $\seqinfn{e_n}$ is a
countable orthonormal basis, then for all $u \in H$, 
we have 
\[
\sum_n \abs{\braket{u}{e_n}}^2 = \norm{u}^2
\]
\end{thm}

\begin{proof}
We have $u = \sum_n \braket{u}{e_n} e_n$. This implies that 
\[
\begin{aligned}
\norm{u}^2 
&= \lim_{m \to \infty} \braket{\sum_{n=0}^m \braket{u}{e_n}
e_n}{\sum_{l=0}^m \braket{u}{e_l} e_l} \\
&= \lim_{m \to \infty} \sum_{n=0}^m \sum_{l=0}^m 
\braket{u}{e_n} \bar{\braket{u}{e_l}} \braket{e_n}{e_l} \\
&= \lim_{m \to \infty} \sum_{n=0}^m \abs{\braket{u}{e_n}}^2 \\
&= \suminfn \abs{\braket{u}{e_n}}^2.
\end{aligned}
\]
\end{proof}

\begin{thm}
If $H$ is an infinte dimensional separable Hilbert space, 
then $H$ is isometrically isomorphic to $\ell^2$. 
That is, there exists bijective boudned linear map 
$T : H \to \ell^2$ such that 
for all $u, v \in H$, we have 
\[
\norm{T u}_{\ell^2} = \norm{u}_H
\text{ and }
\braket{T u}{T v}_{\ell^2} = \braket{u}{v}_H.
\]
\end{thm}

\begin{proof}
Since $H$ is separable, there exists an orthonormal basis
$\seqinfn{e_n}$.
For all $u \in H$, the previous theorem gives
\[
\norm{u} = \left( \suminfn \abs{\braket{u}{e_n}}^2 \right)^{\frac{1}{2}}.
\]
Define $T : H \to \ell^2$ by 
\[
T u = \seqinfn{\braket{u}{e_n}} \in \ell^2.
\]
It is easy to check that $T$ is the desired isometric 
isomorphism.
\end{proof}

Next we use the theories we learned in a more concrete 
setting --- the Fourier series.

\begin{thm}
The subset $\left\{ \frac{e^{i n x}}{\sqrt{2 \pi}} \right\}_{n \in \Z}$
is an orthonormal subset of $L^2([-\pi, \pi])$.
\end{thm}

\begin{defi}
Let $f \in L^2([- \pi, \pi])$. Then the \textbf{$n$-th 
Fourier coefficient} of $f$ is 
\[
\hat{f}(n) = \frac{1}{2\pi} \int_{-\pi}^\pi f(t) 
e^{- i n t} \d t.
\]
The \textbf{$N$-th Fourier sum} of $f$ is 
\[
S_n f(x) = \sum_{\abs{n} \leq N} \hat{f}(n) e^{inx} 
= \sum_{\abs{n} \leq N} \braket{f}{\frac{e^{int}}{\sqrt{2\pi}}}
\frac{e^{inx}}{\sqrt{2\pi}}.
\]

The \textbf{Fourier series} of $f$ is the formal 
series $\sum_{n \in \Z} \hat{f} (n) e^{-inx}$.
\end{defi}

The natural question now is whether we have for 
all $f \in L^2 ([-\pi, \pi])$, 
\[
f(x) = \sum_{n \in \Z} \hat{f}(n) e^{i n x}.
\]
That is, whether we have the following convergence in $L^2$.
\[
\lim_{N \to \infty} \norm{f - S_N f}_2 = 0.
\]
This question is then equivalent to whether 
$\left\{ \frac{e^{inx}}{\sqrt{2 \pi}} \right\}_{n \in \Z}$
is maximal in $L^2 ([-\pi, \pi])$. That is, 
whether $\hat{f}(n) = 0$ for all $n \in \N$ implies $f = 0$.

The answer to the question is yes, but it is going to take 
some work. We first do some simple calculation.

\begin{thm}
For all $f \in L^2([-\pi, \pi])$ and for all $N \in \N$, 
we have 
\[
S_N f(x) = \int_{- \pi}^\pi D_N(x - t) f(t) \d t,
\]
where
\[
D_N(x) = \begin{cases}
  \frac{2N + 1}{2 \pi} & (x = 0) \\
  \frac{\sin \left( N + \frac{1}{2} \right) x}{2 \pi 
  \sin \frac{x}{2}} & (x \neq 0)
\end{cases}
\]
it the \textbf{Dirichlet kernel}. Note that 
$D_N$ is a smooth function.
\end{thm}

\begin{proof}
If $f \in L^2([- \pi, \pi])$, we have 
\[
\begin{aligned}
S_N f(x) 
&= \sum_{\abs{n} \leq N} \left( \frac{1}{2\pi} 
\int_{- \pi}^\pi f(t) e^{- i n t} \d t \right) e^{i n x} \\
&= \int_{-\pi}^\pi f(t) \left( \frac{1}{2\pi} 
\sum_{\abs{n} \leq N} e^{i n (x - t)} \right) \d t.
\end{aligned}
\]
Let $D_N(x) = \frac{1}{2\pi} 
\sum_{\abs{n} \leq N} e^{i n (x - t)}$. 
Then for $x \neq 0$, we have 
\[
\begin{aligned}
  D_N(x) 
  &= \frac{1}{2\pi} \sum_{n=-N}^{N} 
  e^{- i n x} \\ 
  &= \frac{1}{2\pi} e^{- i N x} \sum_{n=0}^{2N}
  \left( e^{i x} \right)^n \\
  &= \frac{1}{2\pi} e^{- i N x} 
  \frac{1 - e^{i (2N+1) x}}{1 - e^{i x}} \\
  &= \frac{1}{2\pi} \frac{e^{i (N + \frac{1}{2}) x} 
  - e^{- i (N + \frac{1}{2}) x}}{e^{\frac{ix}{2}} - 
  e^{-\frac{ix}{2}}} \\
  &= \frac{1}{2\pi} \frac{2 i \sin (N + \frac{1}{2})x}
  {2i \sin \frac{x}{2}} \\
  &= \frac{1}{2\pi} \frac{\sin (N + \frac{1}{2})x}
  {\sin \frac{x}{2}},
\end{aligned}
\]
as desired. For $x = 0$, we also clearly have 
$D_N(0) = \frac{(2N+1)}{2\pi}$. The proof is thus 
complete.
\end{proof}

\begin{defi}
If $f \in L^2([-\pi, \pi])$, we define the 
\textbf{$N$-th Cesaro-Fourier mean} of $f$ by 
\[
\sigma_N f(x) = \frac{1}{N+1} \sum_{k=0}^N 
S_N f(x).
\]
\end{defi}

The idea behind defining the Cesaro mean is that 
if the original sequence converges, the Cesaro mean 
also converge to the same limit. However, Cesaro 
have even better property --- the Cesaro mean 
can converge even if the original sequence does not 
converge. Therefore, it has better convergence properties
and hopefully we can show it converge to $f$ in $L^2$ more 
easily. The goal now is then to show 
\[
\norm{\sigma_N f - f}_2 \to 0 \text{ as $N \to \infty$}.
\]
This would tell us if all Fourier coefficients are zero, 
then the Cesaro means are zero, and the limit above 
would tell us $f$ is zero.

\subsection{Fejer's theorem and Convergence 
of Fourier series}

\end{document}