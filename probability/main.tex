\documentclass[a4paper]{article}
\usepackage{parskip}
\usepackage{lipsum}
\usepackage{eucal}

\def\nterm {Fall}
\def\nyear {2025}
\def\ncourse {Probability}

\input{../header}

\newcommand{\TODO}{\textcolor{red}{\textbf{*** TO-DO ***}}}

\begin{document}
\maketitle

\tableofcontents

\section{Measure theory review}

\subsection{Measurable space and mapping}

\begin{defi}[atom]
Let $\Sigma$ be a $\sigma$-field. Say $A \in \Sigma$ is an atom 
if for all $B \in \Sigma$ either $A \subset B$ or $A \cap B = \emptyset$.
\end{defi}

\begin{prop}
For all $\omega \in \Omega$, there exists atom $A \in \Sigma$ containing 
$\omega$ if $\Omega$ is finite or countable.
\end{prop}

\begin{proof}
  Only prove this for $\Omega$ finite. 
  Define $\tilde{A} = \bigcap \left\{ B \in \Sigma : 
  \omega \in B \right\}$.
\end{proof}

\begin{cor}
If $\Omega$ is finite or countable, there exists a partition 
$\Omega = \bigsqcup_i \Omega_i$,
where each $\Omega_i$ is an atom of $\Sigma$. With this partition, 
$\Sigma$ is just the power set with respect to $\left\{ \Omega_i \right\}_i$.
\end{cor}

\begin{defi}
If $F \subset 2^\Omega$, then the $\sigma$-field generated by $F$ is the 
smallest $\sigma$-field containing all elements of $F$.
\end{defi}

\begin{eg}
Let $\Omega = \left\{ 1, 2, 3, 4, 5 \right\}$ and 
$F = \left\{ \left\{ 2,3 \right\}, \left\{ 3,4 \right\} \right\}$.
Construct $\sigma$-field $\Sigma$ generated by $F$. 
$\Sigma$ is all possible union of sets from the collection 
$\left\{ \{2\}, \{3\}, \{4\}, \{1, 5\}\right\}$.
\end{eg}

\begin{defi}[measurable mapping]
Given two measurable spaces $(\Omega, \Sigma)$ and $(\tilde{\Omega}, 
\tilde{\Sigma})$. Then $f : \Omega \to \tilde{\Omega}$ is measurable 
if $f^{-1} (B) \in \Sigma$ for all $B \in \tilde{\Sigma}$.
\end{defi}

\begin{defi}[Borel $\sigma$-field]
Let $(T, \tau)$ be a topological space. Then the Borel $\sigma$-field 
$\B(T, \tau)$ is defined as the smallest $\sigma$-field containing 
all open sets.
\end{defi}

\begin{defi}[product measurable space]
Given two measurable spaces $(\Omega, \Sigma)$ and 
$(\tilde{\Omega}, \tilde{\Sigma})$. We can define the product 
measurable space as follows: let the ground set be 
$\Omega \times \tilde{\Omega}$, and let 
$\Sigma \otimes \tilde{\Sigma}$ be the smallest $\sigma$-field 
containing all rectangles $B \times \tilde{B}$ where 
$B \in \Sigma$ and $\tilde{B} \in \tilde{\Sigma}$.

More generally, 
let $\Lambda$ be an index set and $(\Omega_\lambda, \Sigma_\lambda)_{\lambda 
\in \Lambda}$. Define the product $\sigma$-field $\bigotimes_{\lambda 
\in \Lambda} \Sigma_\lambda$ be 
the smallest $\sigma$-field containing all elements 
in the form of $\prod_{\lambda \in \Lambda} B_\lambda$ where 
$B_\lambda \in \Sigma_\lambda$ and $B_\lambda = \Omega_\lambda$ for all 
but countably many indices.
\end{defi}

\begin{prop}
  Let $\left( \Omega_i, \Sigma_i \right)_{i=1}^n$ be measurable 
  spaces and $\left( \prod_{i=1}^n \Omega_i, \bigotimes_{i=1}^n \Sigma_i
   \right)$ be the product space. Let $(\Omega, \Sigma)$ be the domain 
  and $f = (f_1, \dots, f_n) : (\Omega, \Sigma) \to (\prod_{i=1}^n 
  \Omega_i, \bigotimes_{i=1}^n \Sigma_i)$. Suppose $f$ is measurable,
  then every coordinate projection $f_i : \Omega \to \Omega_i$ is 
  measurable.

  This is also true for arbitrary index set.
\end{prop}

\subsection{Measure space}
\begin{defi}[measure]
  Let $(\Omega, \Sigma)$ be a measurable space. Then $\mu : \Sigma \to 
  [0, \infty]$ is a measure if 
  \begin{itemize}
    \item $\mu (\emptyset) = 0$. 
    \item If $A_i \in \Sigma$ is pairwise disjoint then 
    $\mu \left( \cupinfi A_i \right) = \suminfi \mu (A_i)$.
  \end{itemize}
\end{defi}

\begin{prop}[continuity of measure]
  If $A_1 \subset A_2 \subset \dots$ is a nested sequence of elements
  of $\Sigma$ and $\mu$ be any measure on $(\Omega, \Sigma)$. Then 
  \[
  \mu \left( \cupinfi A_i \right) = \lim_{i \to \infty} \mu (A_i).
  \]

  If $A_1 \supset A_2 \supset \dots$ is a nested sequence of elements
  of $\Sigma$ and $\mu (A_n) < \infty$ for some $n$. Then 
  \[
  \mu \left( \capinfi A_i \right) = \lim_{i \to \infty} \mu (A_i).
  \]
\end{prop}

\end{document}