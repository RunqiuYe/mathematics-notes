\documentclass[a4paper]{article}
\usepackage{parskip}
\usepackage{lipsum}
\usepackage{newtxtext}
\usepackage{eucal}

\def\nterm {Spring}
\def\nyear {2026}
\def\ncourse {Probability}

\input{../header}
\author{Notes taken by \nauthor\vspace{5pt}\\ 
Lectures by Konstantin Tikhomirov\vspace{5pt}\\
Carnegie Mellon University}

\newcommand{\TODO}{\textcolor{red}{\textbf{*** TO-DO ***}}}

\begin{document}
\maketitle

\tableofcontents

\section{Measure theory review}

\subsection{Measurable space and mapping}

\begin{defi}[$\sigma$-field]
  A collection of subsets $\Sigma \subset 2^{\Omega}$ is a $\sigma$-field if 
  \begin{itemize}
    \item $\emptyset \in \Sigma$.
    \item If $A \in \Sigma$, then $A^c \in \Sigma$.
    \item If $\seqinfi{A_i} \subset \Sigma$, then $\cupinfi A_i \in \Sigma$.
  \end{itemize}
  The pair $(\Omega, \Sigma)$ is called a measurable space.
\end{defi}

\begin{defi}[atom]
Let $\Sigma$ be a $\sigma$-field. Say $A \in \Sigma$ is an atom 
if for all $B \in \Sigma$ either $A \subset B$ or $A \cap B = \emptyset$.
\end{defi}

\begin{prop}
For all $\omega \in \Omega$, there exists atom $A \in \Sigma$ containing 
$\omega$ if $\Omega$ is finite or countable.
\end{prop}

\begin{proof} 
  Define $\tilde{A} = \bigcap \left\{ B \in \Sigma : 
  \omega \in B \right\}$. We can check that $\tilde{A} \in \Sigma$
  and $\tilde{A}$ is an atom containing $\omega$.
\end{proof}

\begin{cor}
If $\Omega$ is finite or countable, there exists a partition 
$\Omega = \bigsqcup_i \Omega_i$,
where each $\Omega_i$ is an atom of $\Sigma$. With this partition, 
$\Sigma$ is just the power set with respect to $\left\{ \Omega_i \right\}_i$.
\end{cor}

\begin{defi}
If $F \subset 2^\Omega$, then the $\sigma$-field generated by $F$ is the 
smallest $\sigma$-field containing all elements of $F$.
Write this $\sigma$-field as $\sigma (F)$.
\end{defi}

\begin{eg}
Let $\Omega = \left\{ 1, 2, 3, 4, 5 \right\}$ and 
$F = \left\{ \left\{ 2,3 \right\}, \left\{ 3,4 \right\} \right\}$.
Construct $\sigma$-field $\Sigma$ generated by $F$. 
$\Sigma$ is all possible union of sets from the collection 
$\left\{ \{2\}, \{3\}, \{4\}, \{1, 5\}\right\}$.
\end{eg}

\begin{defi}[measurable mapping]
Given two measurable spaces $(\Omega, \Sigma)$ and $(\tilde{\Omega}, 
\tilde{\Sigma})$. Then $f : \Omega \to \tilde{\Omega}$ is measurable 
if $f^{-1} (B) \in \Sigma$ for all $B \in \tilde{\Sigma}$.
\end{defi}

\begin{defi}[Borel $\sigma$-field]
Let $(T, \tau)$ be a topological space. Then the Borel $\sigma$-field 
$\B(T, \tau)$ is defined as the smallest $\sigma$-field containing 
all open sets.
\end{defi}

\begin{defi}[product measurable space]
Given two measurable spaces $(\Omega, \Sigma)$ and 
$(\tilde{\Omega}, \tilde{\Sigma})$. We can define the product 
measurable space as follows: let the ground set be 
$\Omega \times \tilde{\Omega}$, and let 
$\Sigma \otimes \tilde{\Sigma}$ be the smallest $\sigma$-field 
containing all rectangles $B \times \tilde{B}$ where 
$B \in \Sigma$ and $\tilde{B} \in \tilde{\Sigma}$.

More generally, 
let $\Lambda$ be an index set and $(\Omega_\lambda, \Sigma_\lambda)_{\lambda 
\in \Lambda}$. Define the product $\sigma$-field $\bigotimes_{\lambda 
\in \Lambda} \Sigma_\lambda$ be 
the smallest $\sigma$-field containing all elements 
in the form of $\prod_{\lambda \in \Lambda} B_\lambda$ where 
$B_\lambda \in \Sigma_\lambda$ and $B_\lambda = \Omega_\lambda$ for all 
but countably many indices.
\end{defi}

\begin{prop}
  Let $\left( \Omega_i, \Sigma_i \right)_{i=1}^n$ be measurable 
  spaces and $\left( \prod_{i=1}^n \Omega_i, \bigotimes_{i=1}^n \Sigma_i
   \right)$ be the product space. Let $(\Omega, \Sigma)$ be the domain 
  and $f = (f_1, \dots, f_n) : (\Omega, \Sigma) \to (\prod_{i=1}^n 
  \Omega_i, \bigotimes_{i=1}^n \Sigma_i)$. Suppose $f$ is measurable,
  then every coordinate projection $f_i : \Omega \to \Omega_i$ is 
  measurable.

  This is also true for arbitrary index set.
\end{prop}

\begin{prop}
  If $f : (\Omega, \Sigma) \to (\Omega_f, \Sigma_f)$ and 
  $g : (\Omega, \Sigma) \to (\Omega_g, \Sigma_g)$, then the concatenation 
  $(f, g)$ is measurable w.r.t. the product space 
  $(\Omega_f \times \Omega_g, \Sigma_f \otimes \Sigma_g)$.
\end{prop}

\begin{proof}
  Let $A \times B$ be such that $A \in \Sigma_f$ and $B \in \Sigma_g$.
  Then the preimage
  \[
  (f, g)^{-1} (A \times B) = f^{-1} (A) \cap g^{-1} (B) \in \Sigma.
  \]
  By definition, the product $\sigma$-field is generated by rectangles, 
  so the proof is complete.
\end{proof}

\subsection{Measure space}
\begin{defi}[measure]
  Let $(\Omega, \Sigma)$ be a measurable space. Then $\mu : \Sigma \to 
  [0, \infty]$ is a measure if 
  \begin{itemize}
    \item $\mu (\emptyset) = 0$. 
    \item If $A_i \in \Sigma$ is pairwise disjoint then 
    $\mu \left( \cupinfi A_i \right) = \suminfi \mu (A_i)$.
  \end{itemize}
\end{defi}

\begin{prop}[continuity of measure]
  If $A_1 \subset A_2 \subset \dots$ is a nested sequence of elements
  of $\Sigma$ and $\mu$ be any measure on $(\Omega, \Sigma)$. Then 
  \[
  \mu \left( \cupinfi A_i \right) = \lim_{i \to \infty} \mu (A_i).
  \]

  If $A_1 \supset A_2 \supset \dots$ is a nested sequence of elements
  of $\Sigma$ and $\mu (A_n) < \infty$ for some $n$. Then 
  \[
  \mu \left( \capinfi A_i \right) = \lim_{i \to \infty} \mu (A_i).
  \]
\end{prop}

\begin{defi}
  Let $(\Omega, \Sigma, \mu)$ be a measure space. 
  
  Say $\mu$ is $\sigma$-finite 
  if there is a representation $\Omega = \cupinfi \Omega_i$ where 
  $\Omega_i \in \Sigma$ and $\mu (\Omega_i) < \infty$.

  Say $\mu$ is a probability measure if $\mu(\Omega) = 1$.
\end{defi}

\begin{defi}[completion of measure space]
  Let $(\Omega, \Sigma, \mu)$ be a measure space. Let 
  \[
  \tilde{\Sigma} = \left\{ A \cup B : A \in \Sigma, 
  B \subset \Omega, \text{there exists $C \in \Sigma$ with 
  $\mu(C) = 0$ and $B \subset C$} \right\}.
  \]
  We can check $\tilde{\Sigma}$ is a $\sigma$-field. If $\tilde{\mu}$
  is a measure on $(\Omega, \tilde{\Sigma})$ which agrees with $\mu$
  on $\Sigma$, then $(\Omega, \tilde{\Sigma}, \tilde{\mu})$ is called 
  a completion of $(\Omega, \Sigma, \mu)$.
\end{defi}

\subsection{$\pi$-$\lambda$ theorem}

\begin{defi}[$\pi$-system]
  Let $\Omega$ be a set and $\P$ be a collection of subsets of $\Omega$.
  Then $\P$ is a $\pi$-system if it is closed with respect to 
  taking finite intersections. That is, $A, B \in \P$ implies $A \cap B 
  \in \P$.
\end{defi}

\begin{eg}
  On the real line $\R$, both $\P_1 = \left\{ (a, b) : a < b \right\}$ 
  and $\P_2 = \left\{ (-\infty, a] : a \in \R \right\}$ are 
  $\pi$-systems.
\end{eg}

\begin{defi}[$\lambda$-system]
  Let $\Omega$ be a set and $\L$ be a collection of subsets of $\Omega$.
  Say $\L$ is a $\lambda$-system if 
  \begin{itemize}
    \item $\emptyset \in \L$.
    \item $A \in \L$ implies $A^c \in \L$.
    \item for all countable collection of disjoint elements $A_i \in \L$,
    we have $\cupinfi A_i \in \L$.
  \end{itemize}
\end{defi}

\begin{thm}[$\pi$-$\lambda$ theorem]
  Let $\Omega$ be a set, $\P$ be a $\pi$-system and $\L$ be a $\lambda$-system.
  Also suppose $\P \subset \L$, then $\sigma (\P) \subset \L$.
\end{thm}

\begin{proof}
  Let $\ell (\P)$ be the smallest $\lambda$-system on $\Omega$ containing 
  $\P$. The goal is to show that $\ell (\P)$ is a $\sigma$-field.
  We need to show that if $A_i \in \ell (\P)$ for $1 \leq i < \infty$, 
  then $\cupinfi A_i \in \ell (\P)$. Note that
  \[
  \cupinfi A_i = \cupinfi \left( A_i \setminus \cupj^{i - 1} A_i \right),
  \]
  so it suffices to show that $A, B \in \ell (\P)$ implies $A \cap B \in 
  \ell (\P)$.

  For $A \in \ell(\P)$ we define 
  \[
  W_A = \left\{ B \subset \Omega : A \cap B \in \ell (\P) \right\}.
  \] 
  It can be directly verified that $W_A$ is a $\lambda$-system.

  Take $A \in \P$, then for any $B \in \P$ we have $A \cap B \in \P 
  \subset \ell (\P)$. Hence, $\P \subset W_A$ and thus $\ell (\P) \subset W_A$
  for all $A \in \P$, as $\ell (\P)$ is the smallest $\lambda$-system on 
  $\Omega$ containing $\P$.
  Now take $A \in \ell (\P)$, we have $A \in W_B$ 
  for all $B \in \P$. It follows that $A \cap B \in \ell (\P)$ and thus 
  $B \in W_A$. Hence similarly 
  $\ell (\P) \subset W_A$ for all $A \in \ell (\P)$.

  Now for any pair $B, C \in \ell (\P)$, we have $C \in W_B$ and thus 
  $B \cap C \in \ell (\P)$. This completes the proof.
\end{proof}

\subsection{Extension theorems}

\begin{defi}[semi-field]
  A collection of subsets $S \subset 2^{\Omega}$ is a semi-field if 
  \begin{itemize}
    \item $\emptyset \in S$ and $\Omega \in S$.
    \item $A, B \in S$ implies $A \cap B \in S$.
    \item If $A \in S$, then $A^c$ is a finite disjoint union of sets in $S$.
  \end{itemize}
\end{defi}

\begin{thm}[Caratheorody's extension theorem]
  Let $S$ be a semi-field and let $\mu$ be a non-negative function on $S$
  satisfying:
  \begin{itemize}
    \item $\mu (\emptyset) = 0$.
    \item If $A_1, \dots, A_n$ are disjoint and $\cupi^n A_i \in S$, then 
    $\mu (\cupi^n A_i) = \sumi^n \mu (A_i)$.
    \item If $A_1, A_2, \dots$ are such that $\cupinfi A_i \in S$, then 
    $\mu (\cupinfi A_i) \leq \suminfi \mu (A_i)$.
  \end{itemize}
  Then $\mu$ admits a unique extension $\bar{\mu}$ which is a 
  measure on $\bar{S}$, the field (algebra) generated by $S$. Moreover, 
  if $\bar{\mu}$ is $\sigma$-finite then $\bar{\mu}$ admits a unique
  extension $\tilde{\mu}$ to $\sigma (S)$.
\end{thm}

\begin{defi}
  Let $T$ be any set. Write 
  \[
  \R^T = \left\{ \left( \omega_t \right)_{t \in T} : 
  \omega_t \in \R \right\}.
  \]
  Also write $\rcal^T$ as the $\sigma$-field generated by rectangles 
  of the form $\prod_{t \in T} I_t$, where for each $t \in T$, 
  $I_t$ is either a semi-open interval of the form $(a, b]$ 
  with $a < b$ or $I_t = \R$, and $I_t = \R$ for all but finitely 
  many $t \in T$.
\end{defi}

\begin{thm}[Kolmogorov's extension theorem]
  For each finite non-empty subset $J \subset T$,
  let $\mu_J$ be a Borel probability measure in $\R^J$, 
  and assume that the measures $\left( \mu_J \right)_{J \subset T, 
  \abs{J} < \infty}$ are
  compatible, in the sense that whenever $J_1 \subset J_2 \subset T$ 
  with $0 \leq \abs{J_1} \leq \abs{J_2} < \infty$,
  $I_j \subset \R$ with $j \in J_1$ are Borel subsets of $\R$, and
  \[
  \tilde{I_j} = \begin{cases}
    I_j & (j \in J_1) \\
    \R & (j \in J_2 \setminus J_1),
  \end{cases}
  \]
  one has 
  \[
  \mu_{J_2} \left( \prod_{j \in J_2} \tilde{I_{j}} \right)
  = \mu_{J_1} \left( \prod_{j \in J_1} I_j \right).
  \]
  Then there exists a unique probability measure $\mu$ on 
  $(\R^T, \rcal^T)$ consistent with $\left( \mu_J \right)_{J \subset T, 
  \abs{J} < \infty}$. That is, one has 
  \[
  \mu \left( \prod_{t \in T} I_t \right)
  = \mu_J \left( \prod_{j \in J} I_j \right)
  \]
  whenever $J \subset T$ with $\abs{J} < \infty$ and $I_t = \R$ 
  for all $t \notin J$.
\end{thm}

\end{document}