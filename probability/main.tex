\documentclass[a4paper]{article}
\usepackage{parskip}
\usepackage{lipsum}

\def\nterm {Fall}
\def\nyear {2025}
\def\ncourse {Probability}

\input{../header}

\newcommand{\TODO}{\textcolor{red}{\textbf{*** TO-DO ***}}}

\begin{document}
\maketitle

\tableofcontents

\setcounter{section}{1}
\section{Laws of large numbers}

\setcounter{subsection}{1}

\subsection{Weak laws of large numbers}
\begin{thm}[$L^2$ weak law]
Let $X_1, X_2, \dots$ be uncorrelated random variables with $EX_i = \mu$
and $E X_i^2 \leq C < \infty$. If $S_n = X_1 + \dots + X_n$ then 
as $n \to \infty$, $S_n / n \to \mu$ in $L^2$ and in probability.
\end{thm}


\begin{lemma}
  If $Y \geq 0$ and $p > 0$ then 
  \[
  E(Y^p) = \int_0^\infty p y^{p - 1} P(Y > y) \d y.
  \]
\end{lemma}

\begin{thm}[Weak law of triangular arrays]
  For each $n$ let $X_{n, k}$, $1 \leq k \leq n$ be independent. 
  Let $b_n > 0$ with $b_n \to \infty$, and let 
  $\bar{X}_{n, k} = X_{n, k} \ind_{(\abs{X_{n, k}} < b_n)}$.
  Suppose that as $n \to \infty$, 
  \begin{enumerate}
    \item $\sumk^n P(\abs{X_{n, k}} > b_n) \to 0$, and
    \item $b_n^{-2} \sumk^n E \bar{X}_{n,k}^2 \to 0$.
  \end{enumerate}
  If we let $S_n = X_{n, 1} + \dots + X_{n, n}$ and put 
  $a = \sumk^n E \bar{X}_{n, k}$, then
  \[
  \frac{S_n - a_n}{b_n} \to 0 \text{ in probability.}
  \]
\end{thm}

\begin{thm}[Weak law of large numbers]
  Let $X_1, X_2, \dots$ be i.i.d. with 
  \[
  x P(\abs{X_i} > x) \to 0
  \]
  as $x \to \infty$. Let $S_n = X_1 + \dots + X_n$ and let 
  $\mu_n = E(X_1 \ind_{(\abs{X_1} \leq n)})$. Then 
  $S_n / n - \mu_n \to 0$ in probability.
\end{thm}

\begin{thm}
  Let $X_1, X_2, \dots$ be i.i.d. with $E \abs{X_i} < \infty$. 
  Let $S = X_1 + \dots + X_n$ and let $\mu = E X_1$. Then 
  $S_n / n \to \mu$ in probability.
\end{thm}

\subsection{Borel-Cantelli Lemmas}

\begin{defi}
  Let $A_n$ be a sequence of subsets of $\Omega$, define 
  \[
  \begin{aligned}
    \limsup A_n 
    &= \bigcap_{m = 0}^\infty \bigcup_{n = m}^\infty A_n 
    = \left\{ \omega \in A_n \text{ infintely often} \right\}, \\
    \liminf A_n 
    &= \bigcup_{m = 0}^\infty \bigcap_{n = m}^\infty A_n 
    = \left\{ \text{$\omega$ in all but finitely many $A_n$} \right\}. \\
  \end{aligned}
  \]
\end{defi}

\begin{prop}
  We have 
  \[
  \limsupn \ind_{A_n} = \ind_{\limsup A_n}, \quad 
  \liminfn \ind_{A_n} = \ind_{\liminf A_n},
  \]
  and 
  \[
  P(\limsup A_n) \geq \limsup P(A_n), \quad 
  P(\liminf A_n) \leq \liminf P(A_n).
  \]
\end{prop}

\begin{thm}[Borel-Cantelli Lemma]
  If $\suminfn P(A_n) < \infty$, then 
  \[
  P(A_n \text{ i.o.}) = 0.
  \]
\end{thm}

\begin{thm}
  $X_n \to X$ in probability iff for every subsequence $X_{n(m)}$, there 
  is a further subsequence $X_{n(m_k)}$ such that $X_{n(m_k)}$ converges 
  almost surely to $X$.
\end{thm}

\begin{thm}[First strong law of large numbers]
  Let $X_1, X_2, \dots$ be i.i.d. with $E X_i = \mu$ and $E X_i^4 < \infty$.
  If $S_n = X_1 + \dots + X_n$, then $S_n / n \to \mu$ a.s.
\end{thm}

\begin{thm}[Second Borel-Cantelli Lemma]
  If events $A_n$ are independent, then $\suminfn P(A_n) = 1$ implies $
  P(A_n \text{ i.o.}) = 1$.
\end{thm}

\begin{thm}
  If $X_1, X_2, \dots$ are i.i.d. with $E \abs{X_i} = \infty$, then 
  $P(\abs{X_n} \geq n \text{ i.o.}) = 1$. Therefore, if $S_n = X_1
  + \dots + X_n$ then 
  \[
  P \left( \lim_{n \to \infty} \frac{S_n}{n} \text{ exists in $(-\infty, 
  \infty)$} \right) = 0.
  \]
\end{thm}

\begin{thm}
  If $A_1, A_2, \dots$ are pairwise independent and $\suminfn P(A_n)
  = \infty$, then 
  \[
  \frac{\summ^n \ind_{A_m}}{\summ^n P(A_n)} \to 1 \text{ a.s.}
  \]
  as $n \to \infty$.
\end{thm}

\subsection{Strong law of large numbers}

\begin{thm}[Strong law of large numbers]
  Let $X_1, X_2, \dots$ be pairwise independent identically distributed
  random variables with $E \abs{X_i} < \infty$. Let $E X_i = \mu$ and 
  $S_n = X_1 + \dots + X_n$. Then $S_n / n \to \mu$ a.s. 
  as $n \to \infty$.
\end{thm}

\begin{thm}
  Let $X_1, X_2, \dots$ be i.i.d. with $E X_i^+ = \infty$ and 
  $E X_i^- < \infty$. If $S_n = X_1 + \dots + X_n$,
  then $S_n / n \to \infty$ a.s.
\end{thm}

\end{document}